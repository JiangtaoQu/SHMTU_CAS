%*********************************************************************
% template.tex
% 本文件主要保存一些导入的包以及宏定义,一般无需修改!
%*********************************************************************
%  SHMTU Course Thesis
% 上海海事大学 课程论文 LaTeX 模板
% Copyright 2023 Haomin Kong
% 2023/11/07 v1.0 beta 0001
%
% 本文来源于 UCASSthesis
% 上述模板来源于人民大学论文模板,已经做了大幅度删改。
% 版权仍归原作者所有。
%*********************************************************************
% 本模板修复大量 bug ,撰写大量注释,以及优化代码结构
%*********************************************************************

\usepackage{ctex}

\usepackage{amsmath}
\usepackage{float}
\usepackage{cite}
\usepackage{listings}
\usepackage{graphicx}
\usepackage{epstopdf}
\usepackage{booktabs}
\usepackage{multirow}

% 页边距定义
\usepackage{geometry}

\usepackage{appendix}
\usepackage{hyperref}
\usepackage{enumerate}
\usepackage{threeparttable}
\usepackage{fancyhdr}
\usepackage{setspace}
\usepackage[T1]{fontenc}

%Times New Roman字体
\usepackage{mathptmx}

\usepackage{titletoc}
\usepackage{fontspec}
\usepackage{pdfpages}
\usepackage{tikz}
\usepackage{etoolbox}
\usepackage{xcolor}
\usepackage{caption}
\usepackage{array}
\usepackage{amssymb}
\usepackage{zhlipsum}

% 显示代码
\usepackage{listings}
\usepackage{xcolor}

\usepackage{algpseudocode}


% 矩阵虚线
% https://blog.csdn.net/bulubuluzst/article/details/119918708
\usepackage{arydshln}

% https://zhuanlan.zhihu.com/p/32925549
% 并排图插入下面三行
%\usepackage{graphicx}
\usepackage{float}
\usepackage{subfigure}

% 使用enumitem宏包来设置enumerate环境的缩进
\usepackage{enumitem}

% 使用amsmath宏包提供的数学环境和命令
\usepackage{amsmath}

% 流程图
\usepackage{tikz}
\usetikzlibrary{positioning, shapes.geometric}

% 英文字体设置为 Times New Roman
\usepackage{fontspec}
\setmainfont{Times New Roman}

% %设置中文字体
% \let\songti\relax
% \let\heiti\relax
% \let\fangsong\relax
% \let\kaishu\relax

% \newCJKfontfamily\sonti{方正书宋简体}[BoldFont=方正粗宋简体]
% \newCJKfontfamily\heiti{方正黑体简体}[BoldFont=方正正粗黑简体]
% \newCJKfontfamily\fangsong{方正仿宋简体}
% \newCJKfontfamily\kaishu{方正楷体简体}[BoldFont=方正粗楷简体]

% \setCJKmainfont{方正书宋简体}[BoldFont = 方正粗宋简体]
% \setCJKsansfont{方正黑体简体}[BoldFont=方正正粗黑简体]
% \setCJKmonofont{方正仿宋简体}

% 参考文献引用上标
% 请使用\upcite代替\cite
\newcommand{\upcite}[1]{\textsuperscript{\textsuperscript{\cite{#1}}}}

% 插入空白页
\newcommand{\emptypage}{
	\clearpage
	\phantom{s}
	\thispagestyle{empty}
}

% 将 lstlisting 的默认标签从 "Listing" 更改为 "代码示例"
\renewcommand{\lstlistingname}{代码示例}

% 设置代码高亮风格等其他选项
\definecolor{dkgreen}{rgb}{0,0.6,0}
\definecolor{mauve}{rgb}{0.58,0,0.82}
\definecolor{lightgray}{gray}{0.95}
\lstset{
	basicstyle=\ttfamily,
	% basicstyle=\ttfamily\small,
	numbers=left,
	%numberstyle=\tiny,
	keywordstyle=\color{blue},
	commentstyle=\color{dkgreen},
	stringstyle=\color{mauve},
	breaklines=true,
	breakatwhitespace=true,
	tabsize=2,
	captionpos=b, % 设置标题位置在底部
	frame=single, % 添加单线边框
	framesep=5pt, % 边框与代码之间的间距
	rulecolor=\color{black}, % 边框颜色
	backgroundcolor=\color{lightgray}, % 设置背景颜色
}
