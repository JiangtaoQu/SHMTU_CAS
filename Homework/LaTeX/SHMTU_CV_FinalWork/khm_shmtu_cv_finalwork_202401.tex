% !TEX program = xelatex
%*********************************************************************
%  SHMTU Course Thesis
% 上海海事大学 课程论文 LaTeX 模板
% Copyright 2023 Haomin Kong
% 2023/11/07 v1.0 beta 0001
%*********************************************************************

% 请注意\chapter只能用于 book 或者 report,不能用于 article
%\documentclass[a4paper,12pt,UTF8]{ctexart}

\documentclass[UTF8,a4paper,12pt,titlepage,oneside]{ctexbook}

% 导入包以及一些定义
%*********************************************************************
% template.tex
% 本文件主要保存一些导入的包以及宏定义,一般无需修改!
%*********************************************************************
%  SHMTU Course Thesis
% 上海海事大学 课程论文 LaTeX 模板
% Copyright 2023 Haomin Kong
% 2023/11/07 v1.0 beta 0001
%
% 本文来源于 UCASSthesis
% 上述模板来源于人民大学论文模板,已经做了大幅度删改。
% 版权仍归原作者所有。
%*********************************************************************
% 本模板修复大量 bug ,撰写大量注释,以及优化代码结构
%*********************************************************************

\usepackage{ctex}

\usepackage{amsmath}
\usepackage{float}
\usepackage{cite}
\usepackage{listings}
\usepackage{graphicx}
\usepackage{epstopdf}
\usepackage{booktabs}
\usepackage{multirow}

% 页边距定义
\usepackage{geometry}

\usepackage{appendix}
\usepackage{hyperref}
\usepackage{enumerate}
\usepackage{threeparttable}
\usepackage{fancyhdr}
\usepackage{setspace}
\usepackage[T1]{fontenc}

%Times New Roman字体
\usepackage{mathptmx}

\usepackage{titletoc}
\usepackage{fontspec}
\usepackage{pdfpages}
\usepackage{tikz}
\usepackage{etoolbox}
\usepackage{xcolor}
\usepackage{caption}
\usepackage{array}
\usepackage{amssymb}
\usepackage{zhlipsum}

% 显示代码
\usepackage{listings}
\usepackage{xcolor}

\usepackage{algpseudocode}


% 矩阵虚线
% https://blog.csdn.net/bulubuluzst/article/details/119918708
\usepackage{arydshln}

% https://zhuanlan.zhihu.com/p/32925549
% 并排图插入下面三行
%\usepackage{graphicx}
\usepackage{float}
\usepackage{subfigure}

% 使用enumitem宏包来设置enumerate环境的缩进
\usepackage{enumitem}

% 使用amsmath宏包提供的数学环境和命令
\usepackage{amsmath}

% 流程图
\usepackage{tikz}
\usetikzlibrary{positioning, shapes.geometric}

% 英文字体设置为 Times New Roman
\usepackage{fontspec}
\setmainfont{Times New Roman}

% %设置中文字体
% \let\songti\relax
% \let\heiti\relax
% \let\fangsong\relax
% \let\kaishu\relax

% \newCJKfontfamily\sonti{方正书宋简体}[BoldFont=方正粗宋简体]
% \newCJKfontfamily\heiti{方正黑体简体}[BoldFont=方正正粗黑简体]
% \newCJKfontfamily\fangsong{方正仿宋简体}
% \newCJKfontfamily\kaishu{方正楷体简体}[BoldFont=方正粗楷简体]

% \setCJKmainfont{方正书宋简体}[BoldFont = 方正粗宋简体]
% \setCJKsansfont{方正黑体简体}[BoldFont=方正正粗黑简体]
% \setCJKmonofont{方正仿宋简体}

% 参考文献引用上标
% 请使用\upcite代替\cite
\newcommand{\upcite}[1]{\textsuperscript{\textsuperscript{\cite{#1}}}}

% 插入空白页
\newcommand{\emptypage}{
	\clearpage
	\phantom{s}
	\thispagestyle{empty}
}

% 将 lstlisting 的默认标签从 "Listing" 更改为 "代码示例"
\renewcommand{\lstlistingname}{代码示例}

% 设置代码高亮风格等其他选项
\definecolor{dkgreen}{rgb}{0,0.6,0}
\definecolor{mauve}{rgb}{0.58,0,0.82}
\definecolor{lightgray}{gray}{0.95}
\lstset{
	basicstyle=\ttfamily,
	% basicstyle=\ttfamily\small,
	numbers=left,
	%numberstyle=\tiny,
	keywordstyle=\color{blue},
	commentstyle=\color{dkgreen},
	stringstyle=\color{mauve},
	breaklines=true,
	breakatwhitespace=true,
	tabsize=2,
	captionpos=b, % 设置标题位置在底部
	frame=single, % 添加单线边框
	framesep=5pt, % 边框与代码之间的间距
	rulecolor=\color{black}, % 边框颜色
	backgroundcolor=\color{lightgray}, % 设置背景颜色
}

\usepackage{ctexsize,type1cm}
\newcommand{\yihao}{\fontsize{26pt}{39pt}\selectfont}
\newcommand{\xiaoyi}{\fontsize{24pt}{36pt}\selectfont}
\newcommand{\erhao}{\fontsize{22pt}{33pt}\selectfont}
\newcommand{\xiaoer}{\fontsize{18pt}{27pt}\selectfont}
\newcommand{\sanhao}{\fontsize{16pt}{24pt}\selectfont}
\newcommand{\xiaosan}{\fontsize{15pt}{22.5pt}\selectfont}
\newcommand{\sihao}{\fontsize{14pt}{21pt}\selectfont}
\newcommand{\xiaosi}{\fontsize{12pt}{18pt}\selectfont}
\newcommand{\wuhao}{\fontsize{10.5pt}{15.75pt}\selectfont}
\newcommand{\xiaowu}{\fontsize{9pt}{13.5pt}\selectfont}
\newcommand{\liuhao}{\fontsize{7.5pt}{11.25pt}\selectfont}

% 格式
% 图表目录重定义
% https://blog.csdn.net/ningxuanyu5854/article/details/80307707
% 为图片添加章节号,不需要请注释下面两行!
\renewcommand {\thetable} {\thesection{}.\arabic{table}}
\renewcommand {\thefigure} {\thesection{}.\arabic{figure}}

% 公式编号添加章节
\numberwithin{equation}{section}
%\numberwithin{subfigure}{section}

%\CTEXsetup[number={\chinese{section}}]{section}
%\CTEXsetup[name={(,)},number={\chinese{subsection}}]{subsection}
%\CTEXsetup[number={\arabic{subsubsection}}]{subsubsection}

% 字体
\usepackage{CJKutf8}
\setCJKfamilyfont{SimHei}{simhei.ttf}
\newcommand{\SimHei}{\CJKfamily{SimHei}}
% Windows 黑体
%\newcommand{\SimHei}{\fontspec{SimHei}}
% macOS  黑体
%\newcommand{\SimHei}{\fontspec{STHeitiSC-Light}}

% 字号
% https://blog.csdn.net/weixin_44026026/article/details/104096778
% https://zhuanlan.zhihu.com/p/464244924
% TODO:标题格式设置
\ctexset{
	% 修改 section。
	section={
	  %name={,.},
	  %number={\chinese{section}},
	  %number={\arabic{section}},
	  format=\SimHei\heiti\zihao{4} % 设置 section 标题为黑体、4号字
	},
	% 修改 subsection。
	subsection={
			%name={,.},
			%number={\arabic{subsection}},
			format=\SimHei\heiti\zihao{4} % 设置 subsection 标题为黑体、4号字
	},
	% 修改 subsubsection。
	subsubsection={
			%name={,.},
			number={\arabic{chapter}.\arabic{section}.\arabic{subsection}.\arabic{subsubsection}},
			format=\SimHei\heiti\zihao{4} % 设置 subsubsection 标题为黑体、4号字
	}
}
% 让目录支持subsubsection
\setcounter{tocdepth}{3}
% 让subsubsection前面正确显示标号
\setcounter{secnumdepth}{3}
%\renewcommand{\thesubsubsection}{\arabic{chapter}.\arabic{section}.\arabic{subsection}.\arabic{subsubsection}}

% 设置目录以及标题的阿拉伯数字的字体
%\renewcommand\thesection{{\fontspec{SimHei}\arabic{section}}}


%页边距设置示例:
% 上(T):2 cm
% 下(B):2 cm
% 左(L):1.5 cm
% 右(R):1.5 cm
% 装订线(T):0.5 cm
% 装订线位置(T):左
%\geometry{a4paper,left = 2cm,right = 1.5cm,top = 2cm,bottom= 2cm}

% TODO:页边距设置
\geometry{a4paper, left = 3cm, right = 2.5cm, top = 3cm, bottom = 3cm}

% 页眉:学校Logo:
% 页码采用10.5号宋体字,居中放置,格式为:第1页。
% 设置页面样式为fancy
\pagestyle{fancy}
% 清空左页眉内容
\lhead{}
%\chead{\includegraphics[width = 4.13cm,height = 0.98cm]{head.png}}
% 清空右页眉内容
\rhead{}
% 清空左页脚内容
\lfoot{}
% 清空右页脚内容
\rfoot{}

% 目录和引用超链接
\hypersetup{
	colorlinks=true,
	linkcolor=black,
	citecolor=black
}

\begin{document}

\newcommand\dunderline[3][-1pt]{{%
			\setbox0=\hbox{#3}
			\ooalign{\copy0\cr\rule[\dimexpr#1-#2\relax]{\wd0}{#2}}}}

% 标题页
\begin{titlepage}

  % 重新定义页面的页边距
  \newgeometry{left=2cm, right=2cm, top=2cm, bottom=2cm}



  \vspace*{2mm}
  \centering

  % TODO:设置LOGO的尺寸
  % 彩色 logo
  \includegraphics[width=0.4\linewidth]{Template/Shanghai_Maritime_University_Logo.eps}
  % 黑白 logo
  %\includegraphics[width=0.4\linewidth]{Template/shmtu_logo_black.eps}

  %\vspace*{-3mm}
  \vspace*{5mm}


  \zihao{1}\textbf{\heiti{上海海事大学课程设计}}

  \vspace{5mm}

  \zihao{1}\textbf{\heiti 基于经典CNN-ResNet的\\上海海事大学统一认证平台\\验证码识别研究及其部署}

  \vspace{3mm}

  %  \begin{spacing}{1.2}
  %    \LARGE\selectfont{\textbf{\heiti ——我是小标题}}
  %  \end{spacing}

  \vspace{5mm}

  % \flushleft
  \begin{flushleft}


    \begin{spacing}{1.3}
      \SimHei

      \hspace{27mm}\heiti\LARGE\selectfont{\textbf{课程名称:}\dunderline[-10pt]{1pt}{\makebox[78mm][c]{{\CJKfamily{SimHei}机器视觉}}}}

      \hspace{27mm}\heiti\LARGE\selectfont{\textbf{任课教师:}\dunderline[-10pt]{1pt}{\makebox[78mm][c]{{\CJKfamily{SimHei}王建华}}}}

      \hspace{27mm}\heiti\LARGE\selectfont{\textbf{学\hspace{14mm}院:}\dunderline[-10pt]{1pt}{\makebox[78mm][c]{物流科学与工程研究院}}}

      \hspace{27mm}\heiti\LARGE\selectfont{\textbf{专\hspace{14mm}业:}\dunderline[-10pt]{1pt}{\makebox[78mm][c]{控制科学与工程}}}

      \hspace{27mm}\heiti\LARGE\selectfont{\textbf{学\hspace{14mm}号:}\dunderline[-10pt]{1pt}{\makebox[78mm][c]{202330510052}}}

      \hspace{27mm}\heiti\LARGE\selectfont{\textbf{学生姓名:}\dunderline[-10pt]{1pt}{\makebox[78mm][c]{{\CJKfamily{SimHei}孔昊旻}}}}

      \hspace{27mm}\heiti\LARGE\selectfont{\textbf{导\hspace{14mm}师:}\dunderline[-10pt]{1pt}{\makebox[78mm][c]{{\CJKfamily{SimHei}李朝锋}}}}

      \hspace{27mm}\heiti\LARGE\selectfont{\textbf{完成日期:}\dunderline[-10pt]{1pt}{\makebox[78mm][c]{2024.02.18}}}

    \end{spacing}

  \end{flushleft}

  \vspace{25mm}

  % 恢复原来的页面页边距
  \restoregeometry

\end{titlepage}

% 封皮后面不是摘要,而是空白页!
\emptypage


%\makedeclare

% 摘要
% 中文摘要
\newpage
\renewcommand{\abstractname}{\textbf{\Large \heiti 摘要}}
% book 与 article 在这里需要切换一下!
%\renewenvironment{abstract}{%
    \newenvironment{abstract}{%
    \par\small
    \noindent\mbox{}\hfill{\bfseries \abstractname}\hfill\mbox{}\par
    \vskip 2.5ex}{\par\vskip 2.5ex}
%中文摘要及关键词放在扉页一、外文摘要及关键词放在扉二,页码编排为Ⅰ,Ⅱ,设置页眉
\begin{abstract}
    %1.5倍行距
    \begin{onehalfspace}

        暂无。

        %\\[12pt]
        \textbf{\textbf{\heiti 关键词:}}上海海事大学
    \end{onehalfspace}
\end{abstract}
\chead{上海海事大学-期末总结}
\setcounter{page}{1}
\pagenumbering{Roman}
\cfoot{\footnotesize \thepage}

% 背面留空
\emptypage

% 中文摘要和英文摘要在同一页,间隔行数
\vspace{5\baselineskip}

% 英文摘要
% 英文摘要另起一页
\newpage

\newcommand{\enabstractname}{\textbf{\Large Abstract}}
\newenvironment{enabstract}{%
    \par\small
    \noindent\mbox{}\hfill{\bfseries \enabstractname}\hfill\mbox{}\par
    \vskip 2.5ex}{\par\vskip 2.5ex}
\begin{enabstract}
    \begin{doublespace}

        None

        %\\[12pt]
        \textbf{Keywords:}ShangHai Maritime University;
    \end{doublespace}
\end{enabstract}
% 英文摘要和中文摘要在同一页时选择一个页眉
%\chead{Shanghai Maritime University Course Design}
\cfoot{\thepage}

% 背面留空
\emptypage

% 目录
\newpage
%\SimHei
%\setmainfont{SimHei}
%\setsansfont{SimHei}
%\setmonofont{SimHei}
\chead{上海海事大学课程设计}
\setcounter{page}{1}
\pagenumbering{Roman}
\cfoot{\footnotesize \thepage}

\renewcommand\contentsname{目\ 录}
\renewcommand\listfigurename{插\ 图}
\renewcommand\listtablename{表\ 格}
%\SimHei
\tableofcontents
\listoffigures
%\listoftables

% 重置起始数字,以及定义样式
\newpage
\setcounter{page}{1}
\setcounter{table}{0}
\setcounter{figure}{0}
\pagenumbering{arabic}
\pagestyle{fancy}
% TODO:页眉设置
\chead{上海海事大学课程设计}
% TODO:页脚的页码格式
\cfoot{\footnotesize \thepage }


% 按照自然段依次排列,每段起行空两格,回行顶格。12号宋体字,(重点文句,12号宋体字,加粗),1.5倍行距。
%正文、注释双面打印,编排页码,自第1页起,设置页眉。

% TODO:设置行距
\begin{spacing}{1.5}

	% 正文
	% TODO:正文字号设置(请参考MyZiHao.tex)
	\xiaosi
	%%%%%%%%%%%%%%%%%%%%%%%%%%%%%%%%%%%%%%%%%%%%%%%%%%%%%%%

	% 将表格计数器的值设置为0,表示从当前位置重新开始对表格进行编号。
	\setcounter{table}{0}
	% 将图片计数器的值设置为0,表示从当前位置重新开始对图片进行编号。
	\setcounter{figure}{0}
	% 以上命令通常在文档中的某个位置使用,以重新开始对表格和图片的编号。
	% 这在需要重新排列表格和图片的顺序或在文档的某个特定部分重新开始编号时非常有用。

	\chapter{背景及意义}
\label{chapter:1}

\section{背景}

2023年秋季,我踏入了上海海事大学的校园,感受着这所学术殿堂的独特氛围。在日常的校园生活中,微信小程序“上海海事大学一卡通”成为了我管理个人账务的得力助手。这款小程序界面简洁明了,功能齐全,尤其是最下方的三个选项卡中的“记录”选项,它详尽地展示了我的每一笔消费记录,如图\ref{fig:wechatbill}所示。这样的设计不仅方便我随时掌握自己的消费情况,还激发了我进一步探索记账领域的兴趣。

然而,在尝试通过网页平台进行更深入的账务管理时,我遇到了一些技术上的挑战。由于平台采用了统一认证接口进行用户身份验证,我的Selenium自动化程序虽然能够自动填充登录信息,但却无法自动处理验证码识别这一环节。这意味着每次运行爬虫程序时,浏览器窗口都会以非headless模式(即用户界面可见)打开,并等待我手动输入验证码。这样的操作流程不仅繁琐,而且无法实现全自动化,给我的账务管理带来了一定的不便。

为了解决这个问题,我开始探索各种可能的解决方案。我研究了验证码识别的相关技术,尝试了多种图像处理和机器学习的方法,以期能够开发出一种能够自动识别验证码的程序。虽然这个过程充满了挑战,但我相信通过不断的学习和实践,我一定能够克服这些困难,实现账务管理的全自动化。同时,我也期待着在未来的学习和工作中,能够将这些技术和经验应用到更广泛的领域,为校园生活和社会发展贡献自己的力量。

\begin{figure}
	\centering
	\includegraphics[width=0.6\linewidth]{Resources/Picture/wechat_bill}
	\caption{微信查询校园卡消费账单}
	\label{fig:wechatbill}
\end{figure}

2024年1月,学期末的考试如期而至,我也在紧张的备考后迎来了一个小小的间歇。应导师的深情“建议”,我决定留校继续学习一个月,以巩固所学知识并进一步深化研究。那段时间,我每天都沉浸在实验室的研究中,常常忙碌到深夜才踏着月色回到宿舍。

有一天,我结束实验回到宿舍时已经夜深人静,时钟的指针早已越过了12点。疲惫不堪的我打算洗个热水澡放松一下身心,然而却发现洗澡水的温度并不高,甚至可以说是有些凉意。在那样一个气温零下的寒夜,这无疑是一种令人沮丧的体验。

我忍不住向师兄抱怨起这件事情,他听后微笑着拿出手机,打开了“海大后勤”微信公众号。他告诉我,这个公众号不仅提供了校园内的各种生活服务信息,还能实时查看每栋宿舍楼当前的洗澡水温度。我好奇地凑过去看,只见屏幕上显示着一个清晰的图表,详细标注了各个宿舍楼的洗澡水温度数据,如图\ref{fig:wechathotwater}所示。

师兄还告诉我,如果我想获取更详细的数据或者进行一些个性化的设置,可以通过抓取相关的URL接口来实现。这立刻引起了我的兴趣,作为一名技术爱好者,我决定尝试一下。我拿出自己的小米手机,利用Root权限开始了抓包操作。经过一番努力,我终于成功抓到了所需的URL接口。

然而,就在我准备进一步探索这个接口的功能时,却发现它同样需要进行统一认证才能访问。这意味着我需要先通过学校的身份验证系统才能获得访问权限。虽然有些失望,但我并没有放弃。我相信通过不断的学习和实践,我一定能够找到解决这个问题的方法,让洗澡水温度的查询变得更加便捷和智能化。同时,我也期待着在未来的学习和生活中,能够将这些技术和经验应用到更广泛的领域,为校园生活和社会发展贡献自己的力量。

\begin{figure}
	\centering
	\includegraphics[width=0.6\linewidth]{Resources/Picture/wechat_hot_water}
	\caption{微信查询洗澡热水水温}
	\label{fig:wechathotwater}
\end{figure}

而且,值得一提的是,小米手机从MIUI 12版本开始,在负一屏为用户提供了各种便捷的卡片式信息展示。这些卡片不仅美观实用,而且能够实时更新各种生活信息,如天气、日程、快递等。我深受启发,认为将来完全可以结合这个桌面小组件的功能,进一步方便我们获取洗澡水的温度信息。想象一下,只需轻轻滑动到负一屏,就能一目了然地看到当前宿舍楼的洗澡水温度,这将极大地提升我们的校园生活体验。

说到校园生活,其实我的学术生涯也与之紧密相连。我们课题组的研究方向主要是利用深度学习的方法进行图像的目标检测与增强处理。为了更深入地掌握这一领域的知识,我特意选修了王建华老师的《机器视觉》课程,以及付广华老师的《Python程序设计》和《人工智能》课程。在科研的过程中,我不断地将所学知识应用到实际项目中,这不仅让我对理论知识有了更深刻的理解,也让我在实践中锻炼了自己的动手能力。

现在,我准备利用我所学的知识来解决统一认证接口的最后一道难题——验证码识别。我深知这是一个充满挑战的任务,因为验证码的设计初衷就是为了防止自动化程序的攻击。然而,我也坚信通过不断的学习和实践,我一定能够找到一种有效的方法来突破这个限制。也许我可以尝试利用图像处理和机器学习的技术来自动识别验证码;也许我可以借鉴其他领域的经验和方法来寻找新的突破口。无论如何,我都将全力以赴,为实现全自动化的账务管理贡献自己的力量。

\section{意义}

这个项目不仅是我对王建华老师的《机器视觉》课程和付广华老师的《Python程序设计》和《人工智能》课程知识的综合运用,更代表着我个人在深度学习领域的学术探索与实践,同时也是我多年计算机程序设计经验的一次全面复习。在这个过程中,我重新回顾了多个曾经熟悉的项目,这些项目涉及不同的领域和技术,让我有机会再次巩固和加深了对计算机程序设计的理解。

此外,我还运用了自己制作的LaTeX模板来整理和呈现项目成果,这不仅提高了我的工作效率,也让我更加熟悉和掌握了LaTeX这一强大的排版工具。通过这个项目,我不仅复习了过往的知识和经验,更重要的是,我解决了自己在实际应用中遇到的一些痛点和难题。这些问题的解决,不仅提升了我的技术水平,也让我更加自信和坚定地走向未来的学习和工作。

同时,我深知学术界的理论研究和工业界的实际应用之间存在着巨大的鸿沟。尽管我在理论层面已经有了一定的积累,但要将这些知识应用到实际工作中,还需要掌握更多实践技能。特别是在模型的量化和部署方面,这是我之前未曾涉足的领域,但却是工业界非常关注的重要环节。因此,我希望通过这个项目,能够将我在课堂、科研以及多年程序设计中积累的知识真正应用到实践中,特别是在模型的量化和部署方面取得突破。

模型的量化与部署是将深度学习技术从理论推向实际应用的关键环节。通过这次实践,我将学习并掌握相关的技术和工具,了解工业界的实际需求和应用场景。这将使我更全面地掌握深度学习的全流程,提高我的实践能力和综合素质,为我未来的职业发展打下坚实的基础。

总的来说,这个项目不仅是我对过往学习成果的一次总结,更是我对未来职业发展的一次有力探索。它让我有机会复习和巩固了多年的计算机程序设计经验,解决了我在实际应用中遇到的痛点,同时也让我更加熟悉和掌握了深度学习领域的先进技术和实践方法。我相信,在未来的学习和工作中,我将以更加扎实的知识储备和更加自信的态度,迎接更多的挑战和机遇。

\section{预备知识}

\begin{enumerate}
	\item OpenCV机器视觉图像处理
	\item 深度学习
	\item KMeans聚类算法
	\item Python语言(Python 3.8+)
	\item C++语言(C++20标准)
	\item Microsoft Visual C++(VC)语言
	\item Windows API
	\item C\# 8.0语言
	\item Java SE(1.8)
	\item Kotlin语言
	\item Android
	\item Gradle构建脚本
	\item PyTorch框架
	\item HTML基础
	\item Selenium自动化测试框架
	\item ONNX推理框架
	\item NCNN推理框架
\end{enumerate}

\section{环境要求}

Windows平台以及Android平台都有方案兼容10年前以前的操作系统,因此无需担心兼容性。

\subsection{MSVC版本}

经过我的测试Windows 7(发布于2019年)可以完美运行。
此外VC运行库官网标注支持Vista操作系统,因此XP以及以上版本操作系统理论上能够正常运行。

\subsection{Qt版本}

Qt支持Windows 7以上版本,但是并不支持Windows 7,而且可能需要安装Direct X图形库。

\subsection{.Net版本}

可能是需要至少Windows 10操作系统。

\subsection{Android平台}

最低运行操作系统版本为Android API 21即Android 5.0(Lollipop)。
API 21发布于2014年,恰好为10年前,这个版本使用ART虚拟机代替了Dalvik虚拟机,因此我选择这个版本作为最低版本。

\section{开发(参考)环境}

下面给出我使用的开发环境供参考。

\subsection{Windows}

硬件:
\begin{enumerate}
	\item Intel Core i5-11400
	\item AMD Radeon RX 580 2048SP
	\item NVIDIA Tesla P40
\end{enumerate}

软件:
\begin{enumerate}
	\item Windows 11 23H2
	\item Visual Studio 2022
	\item cl.exe
	\item CMake
	\item TeXLive 2023
	\item Jetbrains PyCharm
\end{enumerate}

\subsection{Linux}

硬件:
\begin{enumerate}
	\item Intel Core i5-11400
	\item AMD Radeon RX 580 2048SP
	\item NVIDIA Tesla P40
\end{enumerate}

软件:
\begin{enumerate}
	\item Ubuntu 23.10
	\item Linux Kernel 6.5.0-17-generic
	\item KDE Plasma 5.27.8
	\item 图形平台:X.Org(X11)
	\item gcc-12
	\item g++-12
	\item Android Studio Hedgehog 2023.1.1 Patch 2
	\item CMake 3.28.1
	\item Android NDK
	\item Jetbrains PyCharm 2023.3.3
	\item Qt 6.6.2
	\item Jetbrains CLion 2024.1 EAP (Nova)
\end{enumerate}

\subsection{macOS}

硬件:
\begin{enumerate}
	\item Intel Core i7-8850H
	\item AMD Radeon Pro RX 560X
\end{enumerate}

软件:
\begin{enumerate}
	\item macOS 14.2.1
	\item Apple LLVM Clang
	\item CMake 3.28.1
	\item Jetbrains PyCharm 2023.3.3
	\item Qt 6.6.2
	\item Jetbrains CLion 2024.1 EAP (Nova)
\end{enumerate}

	\chapter{问题分析}
\label{chapter:2}

\section{验证码图片分析}

\begin{figure}
	\centering
	\includegraphics[width=0.7\linewidth]{Resources/Picture/test1_20240102160004_server}
	\caption{验证码示例}
	\label{fig:test120240102160004server}
\end{figure}

\section{解决方案}

	\chapter{数据集的获取}
\label{chapter:3}

\section{情况简介}

\section{KMeans聚类算法}

KMeans聚类算法是数据科学领域中最受欢迎和广泛使用的无监督学习算法之一。其核心思想是将数据集中的观测值分组,使得同一组(或称为“聚类”)内的观测值在某种度量下尽可能相似,而不同组之间的观测值则尽可能不同。KMeans算法因其简单性、高效性以及易于解释的结果而在各种应用场景中受到青睐。

\subsection{算法原理与步骤详解}

KMeans聚类算法的工作原理可以概括为以下几个步骤:

\begin{enumerate}
	\item {初始化聚类中心}:首先,算法随机选择数据集中的K个点作为初始的聚类中心。这些点代表了初始的K个聚类。K的值通常由用户根据业务需求或数据分析结果来设定。
	\item {分配数据点到最近聚类}:接下来,算法遍历数据集中的每个点,计算该点到所有聚类中心的距离(通常使用欧几里得距离作为度量标准),然后将该点分配给距离最近的聚类中心。这一步完成后,每个数据点都被分配到了一个聚类中。
	\item {重新计算聚类中心}:在分配完所有数据点后,算法重新计算每个聚类的中心。新的聚类中心是其包含的所有数据点的平均值(在多维空间中,这通常是各维度均值的点)。
	\item {迭代优化}:重复步骤2和3,直到聚类中心不再发生显著变化,或者达到预设的最大迭代次数。每一次迭代都会使聚类中心更接近真实的聚类中心,从而优化聚类结果。
	\item {输出结果}:最终,算法输出K个聚类中心以及每个数据点所属的聚类标签。这些聚类中心和标签可以用于进一步的数据分析或业务决策。
\end{enumerate}

\subsection{距离度量的选择}

在KMeans算法中,距离度量是决定聚类效果的关键因素之一。欧几里得距离是最常用的距离度量方式,它衡量了多维空间中两点之间的直线距离。然而,在某些情况下,其他距离度量方式可能更合适。例如,曼哈顿距离考虑了各维度上的绝对差值之和,适用于某些特定类型的数据(如城市街区距离)。余弦相似性则衡量了两个向量的夹角,更多地关注方向而非大小。选择合适的距离度量方式需要根据数据的性质和业务需求来判断。

\subsection{算法优缺点分析}

KMeans聚类算法的优点在于其简单性、高效性以及易于解释的结果。它能够在短时间内处理大量数据,并输出直观易懂的聚类结果。然而,该算法也存在一些明显的缺点。首先,它需要用户预先设定聚类的数量K,这在很多情况下并不容易确定。其次,KMeans对初始聚类中心的选择非常敏感,不同的初始选择可能导致完全不同的聚类结果。此外,该算法假设每个聚类在所有方向上都是均匀的(即球形聚类),这在现实中可能并不总是成立。最后,KMeans算法在处理噪声和异常值时表现不佳,因为它们可能会对聚类中心产生显著影响。

为了克服这些缺点,研究者们提出了许多改进方法。例如,可以通过多次运行算法并选择最佳结果来减轻对初始聚类中心选择的敏感性;可以使用轮廓系数等评估指标来帮助确定最佳的聚类数量K;还可以考虑使用基于密度的聚类方法(如DBSCAN)来处理非球形聚类或噪声数据。

\subsection{算法应用场景}

KMeans聚类算法在各个领域都有广泛的应用。在市场营销中,它可以帮助企业识别不同的客户群体,从而制定更精准的市场策略。在图像处理中,KMeans可以用于图像压缩和图像分割等任务。在文档聚类中,它可以帮助人们自动地将大量文档按照主题或内容进行分组。此外,KMeans还可以用于异常检测、社交网络分析、生物信息学以及推荐系统等领域。总之,作为一种强大且易于使用的无监督学习算法,KMeans在数据挖掘和机器学习中扮演着重要角色。

\section{KMeans聚类算法程序设计}

\subsection{工作流程}

KMeans聚类算法程序设计的工作流程可以概括为以下几个步骤:

\begin{enumerate}[label=\arabic*.]
	\item \textbf{加载图像}:从指定的目录中加载所有图像文件。这一步通过
	load\_images\_from\_directory函数实现,该函数遍历目录中的文件,筛选出以.jpg或.png为扩展名的图像文件,并使用OpenCV库(`cv2.imread`)读取它们。

	\item \textbf{预处理图像}:将加载的图像展平为一维数组,并组合成一个NumPy数组。这是通过preprocess\_images函数完成的,该函数接受一个图像列表,并将每个图像展平为一个长向量。

	\item \textbf{应用KMeans聚类}:使用apply\_kmeans函数将预处理后的图像数据作为输入,并指定聚类的数量。该函数使用scikit-learn库中的KMeans类来执行KMeans聚类算法,并返回每个图像的聚类标签。

	\item \textbf{复制图像到对应聚类文件夹}:根据聚类标签,将图像复制到输出目录中的相应文件夹。这是通过copy\_images\_to\_clusters函数实现的,该函数首先为每个聚类标签创建一个文件夹(如果不存在),然后根据聚类标签将图像从输入目录复制到相应的聚类文件夹。

	\item \textbf{聚类多个目录中的图像}:cluster\_images函数允许用户指定多个输入目录,并为每个目录中的图像执行上述步骤。该函数遍历输入目录列表,对每个目录中的图像进行聚类,并将结果复制到指定的输出目录。
\end{enumerate}

\subsection{代码解析}

详细代码见"src/cluster/cluster\_func.py"

以下是关键函数的详细解析:

\begin{description}
	\item[load\_images\_from\_directory(directory)]
	此函数遍历给定目录中的所有文件,检查它们是否具有.jpg或.png扩展名,并使用OpenCV的imread函数加载它们。加载的图像被添加到一个列表中并返回。

	\item[preprocess\_images(images)]
	此函数接受一个图像列表,并使用NumPy的flatten方法将每个图像转换为一维数组。然后,它将所有展平的图像组合成一个二维NumPy数组,每行代表一个图像。

	\item[apply\_kmeans(images, num\_clusters)]
	此函数使用scikit-learn的KMeans类来执行聚类。它接受预处理后的图像数据和要创建的聚类数作为输入,并返回每个图像的聚类标签。

	\item[copy\_images\_to\_clusters(input\_directory, output\_directory, labels)]
	此函数根据聚类标签将图像从输入目录复制到输出目录中的相应文件夹。它首先为每个唯一的聚类标签创建一个文件夹(如果尚不存在),然后遍历输入目录中的每个文件,并根据其聚类标签将其复制到相应的文件夹。

	\item[cluster\_images(input\_directories, output\_directory, num\_clusters)]
	此函数是程序的入口点,它接受输入目录列表、输出目录和聚类数作为参数。它遍历每个输入目录,并对其中的图像执行加载、预处理、聚类和复制步骤。这样,用户可以对多个目录中的图像进行聚类,并将结果组织在输出目录中的不同文件夹中。
\end{description}

\section{KMeans聚类算法的应用}

KMeans聚类算法在图像处理中发挥着重要的作用,特别是在数学表达式识别的任务中。本章节将详细介绍如何使用KMeans算法进行数学表达式的分段识别。

相关代码请参考\url{Train/SHMTU_CAS_OCR_RESNET/src/cluster}目录。

\subsection{等号分类的实现}

为了有效地提取出数学表达式中的等号部分,我们采取了裁剪原图的方法。具体而言,我们保留了水平方向上最后的30\%的图片内容,而在垂直方向上则不进行裁剪。这样做的原因是,等号通常在数学表达式的末尾出现,且其高度与整个表达式的高度大致相同。通过这样的裁剪操作,我们能够准确地提取出等号及其周围的区域。

接下来,我们对提取出的图片进行KMeans聚类。由于等号在图像中呈现出特定的视觉特征,因此我们可以将聚类类别设置为两类,以区分等号和非等号区域。通过这一步骤,我们能够得到包含等号的图片集合。

\subsection{计算符的分类策略}

在数学表达式中,常见的计算符包括加号、减号、乘号和除号等。为了对这些计算符进行准确的分类,我们考虑将它们分为以下六类:

\begin{enumerate}
	\item 加号(+)
	\item 减号(-)
	\item 乘号(×)
	\item 汉字“加”
	\item 汉字“减”
	\item 汉字“乘”
\end{enumerate}

针对这六类计算符,我们采用KMeans聚类算法进行分类。由于每类计算符在图像中具有独特的视觉特征,KMeans算法能够有效地将它们区分开来。

\subsection{数字分类的挑战与解决方案}

数字分类是数学表达式识别中的一个重要环节。然而,数字在图像中的表现形式各异,尤其是数字“1”,其宽度较窄,可能导致在按比例裁剪原图时出现错误。具体而言,有时候多余的部分可能会被划分到数字“1”的图片中,而KMeans聚类算法本身并不能很好地处理这种情况。

为了解决这个问题,我们采取了以下策略:在聚类后,我们对每一类进行人工筛选和分类。对于数字类别,我们至少挑选出100张具有代表性的图片,并将其送入卷积神经网络(CNN)中进行进一步的分类。通过CNN的分类,我们能够更准确地识别出每个数字。然后,我们再次进行手动分类,此时的任务变得相对简单,因为错误的分类情况已经大大减少。我们只需要将错误的图片挑选出来,并进行修正即可。

\section{聚类后的后续处理}

经过KMeans聚类算法的处理后,我们得到了各个部分的分类结果。接下来,我们对每一类进行人工筛选和分类,确保分类的准确性。对于每一类,我们至少挑选出100张具有代表性的图片,并将其送入CNN中进行进一步的分类。这样做的目的是为了提高分类的准确性,并确保每个类别都具有足够的样本数量。

在CNN分类之后,我们再次进行手动分类。此时,由于之前的聚类处理已经大大减少了分类错误的情况,因此手动分类的任务变得相对简单。我们只需要将少量分类错误的图片挑选出来,并进行修正即可。通过这样的处理流程,我们能够有效地实现数学表达式的分段识别,为后续的数学表达式解析提供便利。

\section{数据集组成}

数据集在 Hugging Face 公开,仓库地址为 \url{https://huggingface.co/datasets/a645162/shmtu_cas_validate_code}。

下面的结果使用 Python 脚本(Train/SHMTU\_CAS\_OCR\_RESNET/count\_dataset.py)统计。

\subsection{等于符号}

等于符号数据集包含了两种类型的标签:等于号(“=”)和汉字“等于”。
具体组成见表\ref{tab:pic_count_equal_symbol}。

\begin{table}[h]
	\centering
	\begin{tabular}{ccc}
		\toprule
		类别编号 & 符号 & 图片数量 \\
		\midrule
		1 & 等于号(“=”) & 26445张 \\
		2 & 汉字“等于” & 26729张 \\
		\midrule
		总计 & & 53174张 \\
		\bottomrule
	\end{tabular}
	\caption{等于号图片数量统计}
	\label{tab:pic_count_equal_symbol}
\end{table}

\subsection{运算符号}

运算符号数据集包含了三种类型的标签:加号(+)、减号(-)和乘号(×),以及对应的汉字标签。
具体组成见表\ref{tab:pic_count_math_symbols}。

\begin{table}[h]
	\centering
	\begin{tabular}{ccc}
		\toprule
		类别编号 & 符号 & 图片数量 \\
		\midrule
		1 & 加号(+) & 1543张 \\
		2 & 汉字“加” & 1049张 \\
		3 & 减号(-) & 1105张 \\
		4 & 汉字“减” & 1019张 \\
		5 & 乘号(×) & 1038张 \\
		6 & 汉字“乘” & 1199张 \\
		\midrule
		总计 & & 6953张 \\
		\bottomrule
	\end{tabular}
	\caption{数学符号图片数量统计}
	\label{tab:pic_count_math_symbols}
\end{table}

\subsection{数字}

数字数据集包含了从0到9的阿拉伯数字标签。
具体组成见表\ref{tab:pic_count_arabic_numbers}。

\begin{table}[h]
	\centering
	\begin{tabular}{ccc}
		\toprule
		类别编号 & 阿拉伯数字 & 图片数量 \\
		\midrule
		1 & 0 & 3368张 \\
		2 & 1 & 2252张 \\
		3 & 2 & 2808张 \\
		4 & 3 & 2162张 \\
		5 & 4 & 2734张 \\
		6 & 5 & 2801张 \\
		7 & 6 & 3324张 \\
		8 & 7 & 3435张 \\
		9 & 8 & 3367张 \\
		10 & 9 & 3916张 \\
		\midrule
		总计 & & 30167张 \\
		\bottomrule
	\end{tabular}
	\caption{阿拉伯数字图片数量统计}
	\label{tab:pic_count_arabic_numbers}
\end{table}

将数据集与验证集按9:1的比例进行划分,使用ResNet网络进行分类,下面的章节将介绍ResNet网络。

	\chapter{模型的搭建}
\label{chapter:4}

\section{ResNet网络}

\subsection{介绍}

ResNet\cite{He_2016_CVPR},即残差网络,是深度学习领域里程碑式的卷积神经网络架构之一。它在2015年由微软研究院的杰出研究者Kaiming He及其团队首次提出,并在当年的ImageNet图像识别挑战赛中大放异彩,一举夺得了冠军。ResNet的成功并非偶然,它背后蕴含着对深度学习本质的深刻理解与创新设计。

ResNet的核心贡献在于引入了残差学习(residual learning)的崭新概念。在传统的深度神经网络中,随着网络层数的不断增加,训练过程中的梯度消失和模型退化问题日益严重,这极大地限制了网络性能的进一步提升。为了解决这一难题,ResNet巧妙地设计了残差块(residual block),这是一种特殊的网络结构单元,能够使得网络在训练过程中更加关注于残差部分的学习。

具体来说,残差块通过引入跳跃连接(skip connection),将输入直接加到卷积层的输出上,从而形成了残差连接。这种设计不仅有效地缓解了梯度消失的问题,使得梯度能够更加顺畅地回流到较早的层,而且还能够避免模型退化现象的发生。换句话说,即使在网络层数非常深的情况下,ResNet依然能够保持良好的性能表现。

正是由于ResNet在残差学习和残差块设计方面的卓越贡献,它才能够训练出更深、更强大的神经网络模型。这一突破性的成果不仅推动了深度学习领域的快速发展,而且为后来的研究者们提供了宝贵的启示和借鉴。如今,ResNet已经成为计算机视觉领域中最受欢迎的神经网络架构之一,被广泛应用于图像分类、目标检测、语义分割等各种任务中。

\subsection{残差学习}

在深度神经网络中,通常我们认为随着网络层数的不断增加,模型的复杂度也会相应提升,理应能够更好地拟合训练数据。然而,实践却表明,当网络层数增加到一定程度后,训练误差反而会出现上升的现象,这是由于梯度消失和模型退化这两个棘手问题所导致的。梯度消失具体是指在反向传播过程中,梯度值逐层递减,变得极其微小,以至于网络权重几乎无法得到有效更新。而模型退化则表现为随着网络层数的加深,网络的性能不仅没有得到提升,反而出现了下降的情况。

为了解决这些问题,ResNet引入了残差学习的思想,为深度学习领域带来了革命性的突破。残差学习的核心思想在于学习输入与输出之间的残差函数,而非直接学习从输入到输出的复杂映射。具体来说,对于一个网络层,设其输入为$x$,输出为$H(x)$,则残差函数定义为$F(x) = H(x) - x$。在ResNet中,网络通过学习这个残差函数$F(x)$,并将其与输入$x$相加,得到最终的输出$H(x) = F(x) + x$。这种设计巧妙地使得网络在训练过程中更加关注于残差部分的学习,从而有效地避免了梯度消失和模型退化问题。通过残差学习,ResNet成功实现了对网络深度的有效扩展,大幅提升了模型的性能。

\subsection{残差块}

为了实现残差学习,ResNet精心设计了一种特殊的网络结构单元,即残差块。残差块由多个卷积层、批量归一化层和ReLU激活函数组成,形成了一个强大的计算单元。在每个残差块的最后,通过引入一个跳跃连接或shortcut connection,将输入直接加到输出上,实现了残差连接。这种跳跃连接的设计巧妙地使得梯度能够直接回流到较早的层,从而有效地缓解了梯度消失问题。同时,残差块中的批量归一化层也有助于加速训练过程并提高模型的泛化能力。

在ResNet中,根据网络深度的不同需求,设计了两种主要的残差块:基本残差块和瓶颈残差块。基本残差块主要由两个3x3的卷积层组成,适用于构建相对较浅的网络结构。而瓶颈残差块则采用了更为复杂的结构设计,包括1x1、3x3和1x1三个卷积层的组合。其中,1x1的卷积层被用于降低和恢复维度,以减少计算量并提高计算效率。这种设计使得瓶颈残差块更加适用于构建更深的网络结构。

\subsection{网络架构}

ResNet的网络架构由多个精心设计的残差块堆叠而成,形成了一个深度强大且高效的网络模型。根据不同的应用场景和性能需求,ResNet有多个变体可供选择,如ResNet-18、ResNet-34、ResNet-50、ResNet-101和ResNet-152等。这些变体主要在网络深度和残差块类型上有所区别,以满足不同任务的需求。例如,对于较为简单的图像分类任务,可以选择相对较浅的ResNet-18或ResNet-34;而对于更复杂的任务如目标检测或语义分割,则可能需要选择更深的ResNet-50、ResNet-101或ResNet-152以获得更好的性能。

除了标准的ResNet架构外,研究者们还不断探索和改进ResNet的设计,提出了许多改进版本。例如,ResNeXt通过引入分组卷积的思想来扩展ResNet的宽度;SE-ResNet则通过引入注意力机制来增强模型的特征表示能力;而EfficientNet则通过一种复合缩放策略来同时优化网络的深度、宽度和分辨率。这些改进版本在保持ResNet核心思想的同时,进一步提升了网络的性能,推动了深度学习领域的发展。

\subsection{应用与影响}

由于ResNet出色的性能和灵活性,它在计算机视觉领域得到了广泛的应用和认可。除了图像分类任务外,ResNet还被成功应用于目标检测、语义分割、人脸识别、姿态估计等多个具有挑战性的任务中。此外,ResNet的思想也被借鉴到其他类型的神经网络设计中,如循环神经网络和自然语言处理等领域,为这些领域的发展带来了新的启示和机遇。

ResNet的成功对深度学习领域产生了深远的影响。它证明了通过精心设计网络结构和引入新的训练策略,我们可以训练出更深、更强大的神经网络来处理复杂的任务。同时,ResNet也为后续的研究提供了宝贵的经验和启示,推动了深度学习领域的进一步发展和创新。如今,ResNet已经成为深度学习领域中最受欢迎的神经网络架构之一,为人工智能的发展做出了重要贡献。

\section{ResNet网络的应用}

ResNet网络因其强大的特征提取能力而广受欢迎,它经常被用作其他复杂网络结构的基础,即所谓的Back Bone。PyTorch深度学习框架为研究者们提供了极大的便利,内置了ResNet网络模型,这使得我们无需从零开始手动搭建网络结构。

\begin{enumerate}
	\item 为了使用PyTorch内置的ResNet模型,我们首先需要从\textit{torchvision.models}模块中导入预训练好的网络模型。
	\item 接下来,我们需要对模型的最后一层全连接层进行调整,以便适应我们的特定分类任务需求。
\end{enumerate}

关于如何使用ResNet网络的具体代码,请参考我的训练脚本:
\url{Train/SHMTU\_CAS\_OCR\_RESNET/src/classify/step/train.py}

\subsection{等于号的分类任务}

在符号识别领域,等于号的分类任务显得相对简单。由于等于号本身只有两类,即汉字“等于”和符号“=”,这两者在视觉形态上存在着明显的差异。汉字“等于”由两个独立的字符组成,而符号“=”则是一条直线。这种明显的形态差异使得我们不需要采用过于复杂的模型来进行分类。

因此,我们选择使用ResNet-18这一轻量级的卷积神经网络来进行这一2分类任务。ResNet-18以其高效的性能和良好的泛化能力在图像分类任务中得到了广泛的应用。通过利用ResNet-18的强大特征提取能力,我们可以有效地从输入的图像中提取出区分汉字“等于”和符号“=”的关键特征,从而实现高精度的分类。

在训练过程中,我们将准备大量的带有标签的等于号图像数据,这些数据将用于训练ResNet-18模型。通过不断地优化模型的参数,我们可以使得模型对于这两类等于号的识别能力逐渐提高。最终,我们可以得到一个能够准确区分汉字“等于”和符号“=”的ResNet-18模型,为后续的符号识别任务提供有力的支持。

\subsection{运算符的分类任务}

运算符的分类在符号识别领域中扮演着至关重要的角色,它涵盖了数学和汉字领域中多种常见的运算符。这不仅仅是一个单纯的图像识别任务,更是一个涉及到语义理解和上下文分析的综合任务。在本分类任务中,我们主要关注以下六类运算符:

\begin{enumerate}
	\item 加号(+):这是数学运算中的基石,表示两个数值的相加。加号在日常生活、学术研究以及各类计算中无处不在,其准确识别对于任何符号识别系统来说都至关重要。
	\item 减号(-):与加号相对应,减号用于表示两个数值之间的差值。在数学表达式和计算过程中,减号的准确识别同样具有不可或缺的地位。
	\item 乘号(×):乘号在数学中用于表示两个数值的相乘。尽管在现代计算中,乘号有时被省略或用星号(*)代替,但乘号作为标准的乘法运算符,其地位仍不可撼动。
	\item 汉字“加”:在中文数学表达式中,汉字“加”被广泛用于表示加法运算。与数学符号加号(+)不同,汉字“加”不仅提供了数学运算的信息,还蕴含了中文的语义和语境,使得数学表达式更易于理解和解读。
	\item 汉字“减”:在中文数学语境下,汉字“减”同样扮演着重要的角色,用于表示减法运算。对于包含减法的数学语句来说,汉字“减”的准确识别是理解整个语句意义的关键。
	\item 汉字“乘”:与数学符号乘号(×)功能相似,汉字“乘”在中文中也被用于表示乘法运算。与数学符号不同,汉字“乘”提供了更多的上下文信息,使得数学表达式在中文语境下更加清晰和易于理解。
\end{enumerate}

这六类运算符,无论是数学符号还是汉字,都在数学和中文表达中占据了举足轻重的地位。它们的准确识别对于构建一个高效、准确的符号识别系统来说至关重要。尽管这六类运算符在形态和语义上各有特点,但考虑到ResNet-18模型在图像分类任务中的出色表现以及其对复杂特征的强大捕捉能力,我们依然选择使用ResNet-18模型进行这六类运算符的分类任务。通过精心设计和调整模型的训练策略,我们有信心能够训练出一个能够准确区分这六类运算符的识别系统,为后续的符号识别任务提供有力的支持。

\subsection{阿拉伯数字的分类任务}

阿拉伯数字的分类任务在符号识别领域中具有举足轻重的地位。与运算符分类任务不同,阿拉伯数字的分类涵盖了从0到9的十个独立类别,每个数字都具有其独特的形态和语义含义。这些数字在日常生活中扮演着至关重要的角色,不仅用于数学计算,还广泛应用于时间表示、文档编号、电话号码等各个领域。

要实现对阿拉伯数字的准确分类,我们需要构建一个精细且稳健的识别系统。这个系统需要具备出色的区分能力,特别是对于那些形态相似的数字,如6和9,它们之间的细微差别往往决定了识别的准确性。此外,由于在实际应用中,可能会遇到裁剪不完整或书写风格各异的数字图像,如数字7在裁剪不完整时可能与带衬线的数字1混淆,这对识别系统提出了更高的要求。

为了应对这些挑战,我们不仅需要收集大量带有标签的阿拉伯数字图像数据,还需要选择一个具有强大特征提取能力的模型。尽管ResNet-18模型在许多任务中表现出色,但由于其对于复杂和细微特征识别的局限性,我们可能需要考虑一个更高级的模型。

在综合考虑模型的性能、体积和准确率之后,我们选择了ResNet-34模型进行阿拉伯数字的分类任务。ResNet-34模型通过其更深的网络结构和更强的特征提取能力,能够更有效地捕捉和区分阿拉伯数字的关键特征,特别是在处理形态相似或裁剪不完整的数字时。同时,我们也对模型进行了优化,以确保在保持较高准确率的同时,尽可能地减小体积和计算复杂度,以满足移动端部署的需求。

通过利用ResNet-34模型进行训练,我们可以期望构建一个既高效又准确的阿拉伯数字识别系统。这将为各种依赖数字识别的应用场景提供强大的支持,如文档自动化处理、表格识别、验证码识别等。同时,这一技术也将为相关领域的研究和开发提供新的思路和方法。

	\chapter{推理与验证}
\label{chapter:5}

\section{推理}

深度学习的推理(Inference)是指利用已经训练好的深度学习模型对新的、未见过的数据进行预测或分类的过程。在推理阶段,模型会接收输入数据,并通过一系列的计算和操作,最终输出预测结果。这些计算和操作通常包括矩阵乘法、激活函数运算、卷积运算等,具体取决于模型的架构和设计。

与训练阶段不同,推理阶段不需要更新模型的权重参数,而是使用已经训练好的权重对输入数据进行前向传播计算。因此,推理过程通常比训练过程更快,更适合于实时应用或大规模数据处理。

在深度学习领域,推理是非常重要的一个环节,它使得深度学习模型能够应用于实际场景中,并为各种任务提供智能化的解决方案。例如,在计算机视觉领域,深度学习模型可以用于图像分类、目标检测、人脸识别等任务;在自然语言处理领域,深度学习模型可以用于文本生成、情感分析、机器翻译等任务。

在数学上,这可以表示为一个函数映射关系:

\[ y = f(x; \theta) \]

其中:

\begin{itemize}
	\item $x$ 是输入数据,可以是一个向量、矩阵或张量,取决于模型的输入层设计。
	\item $\theta$ 是模型的权重参数,这些参数在训练阶段通过优化算法学习得到,并在推理阶段保持不变。
	\item $f$ 是模型的函数映射关系,它描述了输入数据如何通过模型的各层计算得到输出预测结果。这个函数通常是由多层神经网络组成的复杂非线性映射。
	\item $y$ 是模型的输出预测结果,它可以是一个标量、向量或矩阵,取决于模型的输出层设计和任务类型。
\end{itemize}

在推理阶段,模型接收输入数据 $x$,并使用已经训练好的权重参数 $\theta$ 进行前向传播计算。这个过程可以分解为多个步骤,每个步骤对应模型中的一层或一个操作。例如,对于全连接层(Dense Layer)或卷积层(Convolutional Layer),计算可以表示为:

\[ z = Wx + b \]
\[ a = \sigma(z) \]

其中:

\begin{itemize}
	\item $W$ 和 $b$ 分别是该层的权重矩阵和偏置向量。
	\item $z$ 是该层的线性输出。
	\item $\sigma$ 是激活函数,用于引入非线性特性。常见的激活函数包括ReLU、Sigmoid和Tanh等。
	\item $a$ 是该层的激活输出,也是下一层的输入。
\end{itemize}

通过逐层进行这样的计算,最终可以得到模型的输出预测结果 $y$。这个过程通常比训练过程更快,因为它不需要更新权重参数或计算梯度信息。因此,推理过程更适合于实时应用或大规模数据处理。

\section{推理框架介绍}

\subsection{ONNX}

ONNX (Open Neural Network Exchange) 的出现,为深度学习领域带来了革命性的变化。作为一个开放的模型标准,ONNX旨在实现不同深度学习框架之间的互操作性,从而简化模型的部署和推理过程。随着其不断发展和完善,ONNX已经得到了越来越多框架的支持,形成了一个庞大的生态系统。

目前,支持ONNX的深度学习框架数量已经相当可观。除了最初的TensorFlow、PyTorch和MXNet等主流框架外,还有许多其他框架也加入了ONNX的阵营。例如,Caffe2、PyTorch Mobile、Microsoft Cognitive Toolkit (CNTK)、PaddlePaddle、Theano等,都已经提供了对ONNX的支持。这意味着开发者可以在这些框架中训练模型,然后将其导出为ONNX格式,以便在其他框架或平台上进行推理。

ONNX的支持框架数量不断增加,不仅反映了其在深度学习领域的广泛认可和应用,也为开发者提供了更多的选择和灵活性。无论是学术研究还是实际应用,开发者都可以根据自己的需求和偏好选择合适的框架进行模型训练和推理。同时,由于ONNX的通用性和开放性,它也促进了不同框架之间的交流和合作,推动了深度学习技术的进一步发展。

除了支持多种框架外,ONNX还注重与各种硬件平台的兼容性。无论是CPU、GPU还是FPGA等硬件平台,ONNX都可以提供高效的推理性能。这使得开发者可以根据具体的应用场景和性能需求选择合适的硬件进行推理,从而实现更好的性能和能效比。

总的来说,ONNX作为一个开放的模型标准,已经得到了众多深度学习框架的支持和认可。它的出现打破了不同框架之间的壁垒,促进了深度学习技术的广泛应用和发展。随着其不断完善和扩展,相信ONNX将在未来的深度学习领域发挥更加重要的作用。

\subsection{Intel OpenVINO}

Intel OpenVINO(Open Visual Inference \& Neural Network Optimization),作为Intel公司精心打造的深度学习推理和神经网络优化工具套件,自推出以来,便在工业界和学术界引起了广泛的关注和应用。这套工具套件不仅全面支持Intel的多种硬件平台,包括CPU、GPU、FPGA以及专门的神经网络处理器(如Intel Neural Compute Stick和Intel Neural Compute Stick 2),还针对这些硬件进行了深度优化,以确保在各种应用场景下都能实现卓越的推理性能。

OpenVINO的核心优势在于其强大的硬件加速能力。通过充分利用Intel硬件的并行计算、矢量处理和低功耗等特性,OpenVINO能够显著提升深度学习模型的推理速度和能效比。这意味着在相同的硬件条件下,使用OpenVINO进行推理往往能够获得更高的吞吐量和更低的延迟,从而满足各种实时性要求较高的应用场景,如自动驾驶、智能监控、人机交互等。

除了硬件加速外,OpenVINO还提供了一系列软件工具和库来帮助开发者进行模型优化和部署。其中,模型优化器(Model Optimizer)是一个非常重要的组件,它可以将训练好的模型转换为OpenVINO支持的中间表示格式(Intermediate Representation,IR),以便在Intel硬件上进行高效推理。此外,OpenVINO还提供了推理引擎(Inference Engine),它负责加载IR格式的模型并执行推理任务,同时还支持多种输入输出格式和预处理后处理操作,以方便开发者与各种数据源和设备进行集成。

在实际应用中,OpenVINO的易用性和灵活性也得到了广泛认可。它支持多种编程语言和开发环境,包括C++、Python、Java等,同时还提供了丰富的API和文档资源,以帮助开发者快速上手和解决问题。此外,OpenVINO还积极与各种开源框架和社区进行合作,如TensorFlow、PyTorch、ONNX等,以实现无缝的模型导入和转换。这种开放和包容的态度不仅降低了开发者的学习成本和技术门槛,也促进了深度学习技术的广泛应用和发展。

总的来说,Intel OpenVINO是一个功能强大且易于使用的深度学习推理框架。它充分利用了Intel硬件的性能优势,为开发者提供了高效、灵活的推理解决方案。无论是在云端还是在边缘端,无论是在计算机视觉还是在语音识别等领域,OpenVINO都展现出了卓越的性能和广泛的应用前景。随着深度学习技术的不断发展和普及,相信OpenVINO将在未来的智能时代发挥更加重要的作用。

\subsection{腾讯优图NCNN}

腾讯优图实验室的NCNN(Neural Network Inference Framework)是一个专为移动端设计的高效神经网络前向计算框架。在移动设备上实现高性能、轻量级的深度学习模型推理,一直是深度学习领域的一大挑战。NCNN的出现,正是为了应对这一挑战而生。

NCNN的特点在于其极致的性能优化和轻量级的设计。针对移动设备的硬件特性和资源限制,NCNN进行了一系列的优化措施,包括计算图优化、内存管理优化、多线程加速等,以确保在有限的计算资源和内存空间下,仍能实现高效的模型推理。这种优化不仅提升了推理速度,还降低了功耗,延长了移动设备的续航时间。

除了性能优化外,NCNN还注重易用性和兼容性。它提供了丰富的API接口和文档资源,支持多种输入输出格式和预处理后处理操作,以方便开发者与各种数据源和设备进行集成。同时,NCNN还支持多种深度学习模型和框架的导入和转换,如TensorFlow、Caffe等,这大大降低了开发者的迁移成本和学习门槛。

在实际应用中,NCNN已经被广泛应用于各种移动设备和嵌入式系统中。无论是人脸识别、语音识别还是图像分类等任务,NCNN都能提供高效、准确的推理结果。同时,由于其轻量级的设计,NCNN还特别适合于在资源受限的环境下进行部署,如智能家居、无人机等领域。

腾讯优图实验室作为NCNN的开发者,一直在不断完善和扩展这个框架。他们积极与开发者社区进行交流与合作,收集用户的反馈和需求,以便不断改进和优化NCNN的性能和功能。这种开放和包容的态度,使得NCNN在移动端深度学习领域赢得了广泛的认可和支持。

总的来说,腾讯优图NCNN是一个专注于移动端深度学习模型推理的高效框架。它通过极致的性能优化和轻量级的设计,实现了在移动设备上高效、准确的模型推理。随着移动设备和嵌入式系统的普及和发展,相信NCNN将在未来的深度学习领域发挥更加重要的作用。

\subsection{Microsoft Windows ML}

Microsoft Windows ML 是微软为Windows 10及更高版本量身打造的机器学习API,它的出现标志着Windows平台对深度学习和机器学习技术的全面支持。Windows ML不仅为开发者提供了一个简单、高效的方式来集成机器学习功能到他们的应用中,而且还充分利用了Windows操作系统的底层优化和硬件加速能力,从而确保了出色的推理性能和兼容性。

Windows ML的核心优势在于其对ONNX模型的全面支持。ONNX(Open Neural Network Exchange)是一个开放的模型标准,旨在实现不同深度学习框架之间的互操作性。通过支持ONNX,Windows ML允许开发者在Windows应用中使用各种预训练的模型进行推理,无论这些模型最初是在TensorFlow、PyTorch、MXNet还是其他框架中训练的。这种灵活性不仅降低了开发者的迁移成本,还促进了跨平台和跨框架的模型共享与协作。

除了对ONNX的支持外,Windows ML还提供了一系列易于使用的API和工具,以帮助开发者快速集成机器学习功能到他们的应用中。这些API和工具不仅简化了模型的加载、预处理和后处理过程,还提供了对硬件加速和性能优化的深度控制。这意味着开发者可以根据具体的应用场景和性能需求,选择合适的硬件和优化策略来执行推理任务,从而实现更高的吞吐量和更低的延迟。

在实际应用中,Windows ML已经被广泛应用于各种Windows应用中,包括图像识别、语音识别、自然语言处理等。通过与Windows操作系统的紧密集成,Windows ML不仅提升了这些应用的智能化水平,还为用户提供了更加流畅、自然的交互体验。同时,由于其底层优化和硬件加速能力,Windows ML也确保了在这些应用中实现卓越的推理性能和能效比。

总的来说,Microsoft Windows ML是一个功能强大且易于使用的机器学习API,它为Windows开发者提供了一个简单、高效的方式来集成机器学习功能到他们的应用中。通过支持ONNX模型和提供丰富的API和工具,Windows ML不仅降低了开发者的学习成本和技术门槛,还促进了跨平台和跨框架的模型共享与协作。随着Windows操作系统的不断发展和普及,相信Windows ML将在未来的机器学习领域发挥更加重要的作用。

\subsection{NVIDIA TensorRT}

NVIDIA TensorRT是一个专为高性能深度学习推理而设计的优化库。作为NVIDIA深度学习生态系统的重要组成部分,TensorRT充分利用了NVIDIA GPU硬件的并行计算能力和优化技术,为开发者提供了极致的推理性能和能效比。

TensorRT的核心优势在于其针对NVIDIA GPU的深度优化。通过一系列先进的优化技术,如层融合、精度校准、动态张量内存管理、核自动调整等,TensorRT能够显著提升深度学习模型的推理速度。这种优化不仅减少了计算资源和内存占用,还降低了功耗,使得在相同的硬件条件下,使用TensorRT进行推理往往能够获得更高的吞吐量和更低的延迟。

除了硬件优化外,TensorRT还注重易用性和灵活性。它支持多种深度学习框架和模型格式,如TensorFlow、PyTorch、ONNX等,方便开发者将训练好的模型导入到TensorRT中进行优化和推理。同时,TensorRT还提供了丰富的API和工具,以帮助开发者进行模型解析、优化、序列化和部署。这种开放和包容的态度不仅降低了开发者的学习成本和技术门槛,还促进了深度学习技术的广泛应用和发展。

在实际应用中,TensorRT已经被广泛应用于各种需要高性能推理的场景,如自动驾驶、智能监控、语音识别等。通过与NVIDIA GPU的紧密集成,TensorRT不仅提升了这些应用的智能化水平,还为用户提供了更加流畅、自然的交互体验。同时,由于其针对NVIDIA GPU的深度优化,TensorRT也确保了在这些应用中实现卓越的推理性能和能效比。

总的来说,NVIDIA TensorRT是一个功能强大且易于使用的高性能深度学习推理优化库。它通过充分利用NVIDIA GPU的硬件优势和提供一系列优化技术,为开发者提供了极致的推理性能和能效比。随着深度学习技术的不断发展和普及,相信TensorRT将在未来的智能时代发挥更加重要的作用。

\subsection{Apple CoreML}

Apple CoreML是一个强大的机器学习框架,专为苹果公司的iOS、macOS、watchOS和tvOS应用设计。它使开发者能够轻松地将机器学习模型集成到他们的应用中,无论是用于图像识别、自然语言处理、语音识别还是其他复杂的机器学习任务。CoreML的出现,不仅简化了机器学习在苹果设备上的部署流程,还显著提升了应用的智能化水平和用户体验。
`
CoreML支持多种模型类型,包括神经网络、决策树、支持向量机等,几乎涵盖了当前主流的机器学习算法。这意味着开发者可以根据具体的应用场景和需求,选择最合适的模型类型来实现高效、准确的推理。同时,CoreML还支持从多种深度学习框架导入模型,如TensorFlow、Caffe等,这大大降低了开发者的迁移成本和学习门槛。

除了广泛的模型支持外,CoreML还注重推理性能的优化。它充分利用了苹果设备的硬件特性,如CPU、GPU和神经引擎等,通过一系列优化技术来提升推理速度和能效比。这些优化技术包括计算图优化、内存管理优化、硬件加速等,以确保在各种应用场景下都能实现卓越的推理性能。

在实际应用中,CoreML已经被广泛应用于各种苹果设备上。无论是用于人脸识别解锁手机,还是用于智能推荐音乐和视频,CoreML都展现出了出色的性能和稳定性。同时,由于其与苹果生态系统的紧密集成,CoreML还为开发者提供了一系列易于使用的API和工具,以方便他们快速集成机器学习功能到应用中。

总的来说,Apple CoreML是一个功能强大且易于使用的机器学习框架。它通过支持多种模型类型和提供高效的推理性能,为开发者提供了一个简单、高效的方式来集成机器学习功能到苹果设备上。随着机器学习技术的不断发展和普及,相信CoreML将在未来的智能时代发挥更加重要的作用。

\subsection{TensorFlow Lite}

TensorFlow Lite,作为TensorFlow的一个轻量级版本,是专门为移动端和嵌入式设备量身打造的。考虑到这些设备的资源限制和性能特点,TensorFlow Lite在保持TensorFlow强大功能的同时,对模型推理过程进行了深度优化,以确保在有限的计算资源和内存空间下,仍能实现高效、准确的模型推理。

TensorFlow Lite的核心优势在于其对模型的优化能力。它提供了一系列优化工具和技术,包括量化、剪枝等,这些工具和技术可以显著降低模型的大小和计算复杂度,从而提高推理速度并减少功耗。这种优化不仅使得模型更适合在移动端和嵌入式设备上运行,还延长了设备的续航时间,提升了用户体验。

除了优化能力外,TensorFlow Lite还注重易用性和兼容性。它支持从TensorFlow模型直接转换得到,这大大降低了开发者的迁移成本。同时,TensorFlow Lite还提供了丰富的API和工具,以方便开发者进行模型的加载、预处理、推理和后处理等操作。这些API和工具不仅简化了开发流程,还提高了开发效率。

在实际应用中,TensorFlow Lite已经被广泛应用于各种移动端和嵌入式设备中。无论是图像分类、语音识别还是自然语言处理等任务,TensorFlow Lite都能提供高效、准确的推理结果。同时,由于其轻量级的设计和优化能力,TensorFlow Lite还特别适合于在资源受限的环境下进行部署,如智能家居、无人机等领域。

总的来说,TensorFlow Lite是一个功能强大且易于使用的轻量级机器学习框架。它通过一系列优化工具和技术,提高了模型在移动端和嵌入式设备上的推理速度和能效比。随着移动设备和嵌入式系统的普及和发展,相信TensorFlow Lite将在未来的机器学习领域发挥更加重要的作用。

\subsection{Paddle}

Paddle(PArallel Distributed Deep LEarning),由百度公司主导开发,是一个功能全面且高效的深度学习平台。该平台不仅为开发者提供了丰富的模型库和工具,还支持多种应用场景,从图像识别、自然语言处理到语音识别和自动驾驶等,几乎涵盖了当前深度学习的所有热门领域。

Paddle的设计理念始终围绕着易用性、高效性和灵活性。其提供的模型库包含了众多预训练模型,这些模型都是在大规模数据集上经过精心训练和优化得到的,可以直接用于实际任务中,大大降低了开发者的时间和成本。同时,Paddle还提供了一系列强大的工具,如模型可视化、调试工具、性能分析工具等,帮助开发者更加高效地进行模型的开发、训练和调优。

值得一提的是,Paddle还特别关注移动端的深度学习应用。为了满足移动设备上部署深度学习模型的需求,Paddle推出了移动端推理框架——Paddle Lite。Paddle Lite是一个轻量级的推理框架,专门针对移动设备的特性和限制进行了优化。它支持多种移动端硬件平台,包括ARM、x86等,能够充分利用硬件资源来实现高效的推理性能。同时,Paddle Lite还提供了模型压缩、量化等优化技术,进一步减小了模型的大小和提高了推理速度,使得深度学习模型在移动设备上也能实现快速、准确的推理。

在实际应用中,Paddle和Paddle Lite已经被广泛应用于各种场景。无论是百度的搜索引擎、语音识别服务,还是各种移动应用中的图像识别、自然语言处理等功能,背后都有Paddle和Paddle Lite的强大支持。这种广泛的应用不仅验证了Paddle平台的强大功能和高效性能,也展示了深度学习技术在实际问题中的巨大潜力和价值。

\subsection{MACE}

MACE(Mobile AI Compute Engine)是小米公司开源的一个专为移动端设备设计的深度学习推理框架。随着移动设备的普及和计算能力的提升,将深度学习技术应用到移动端已成为一种趋势。MACE应运而生,旨在帮助开发者在移动端设备上实现高效、准确的深度学习推理。

MACE支持多种神经网络模型,包括但不限于卷积神经网络(CNN)、循环神经网络(RNN)和生成对抗网络(GAN)等。这种广泛的模型支持使得开发者可以根据具体的应用场景和需求选择合适的模型进行推理。同时,MACE还支持多种硬件平台,如ARM、x86等,确保了在不同移动设备上的兼容性和高效性。

为了提高推理性能,MACE提供了一系列优化技术。这些技术包括模型压缩、量化、剪枝等,旨在减小模型的大小和降低计算复杂度,从而加快推理速度并减少功耗。通过这些优化技术,MACE能够在保持模型精度的同时,显著提升推理性能,为移动端应用带来更好的用户体验。

除了性能优化外,MACE还注重易用性和可扩展性。它提供了简洁明了的API和丰富的文档,方便开发者快速上手并进行模型推理。同时,MACE还支持自定义扩展,开发者可以根据自己的需求添加新的模型和优化技术,以满足特定的应用场景。

在实际应用中,MACE已经被广泛应用于各种移动端应用中,如图像识别、语音识别、自然语言处理等。通过与小米设备的紧密集成,MACE不仅提升了这些应用的智能化水平,还为用户提供了更加流畅、自然的交互体验。同时,由于其开源的特性,MACE也促进了深度学习技术在移动端的广泛应用和发展。

总的来说,MACE是一个功能强大且易于使用的移动端深度学习推理框架。它通过支持多种神经网络模型和硬件平台,并提供一系列优化技术,为开发者提供了高效、准确的推理性能。随着移动设备计算能力的不断提升和深度学习技术的不断发展,相信MACE将在未来的移动智能时代发挥更加重要的作用。

\subsection{TNN}

TNN(Tiny Neural Network)是由腾讯公司主导研发的一个轻量级的深度学习推理框架。在移动端设备日益普及的今天,TNN的出现为在资源受限的环境下实现高性能、低延迟的深度学习推理提供了可能。

TNN的设计理念始终围绕着轻量级和高性能。为了实现这一目标,TNN在框架设计、模型优化和硬件加速等方面进行了全面考虑。首先,TNN采用了简洁高效的框架设计,去除了不必要的依赖和冗余代码,使得整个框架非常轻巧,易于集成到各种移动端应用中。其次,TNN提供了一系列模型优化技术,如量化、剪枝等,以降低模型的大小和计算复杂度,从而提高推理速度和能效比。最后,TNN还充分利用了移动端设备的硬件特性,如CPU、GPU等,通过硬件加速技术来进一步提升推理性能。

除了轻量级和高性能外,TNN还注重兼容性和易用性。它支持多种神经网络模型格式,如TensorFlow、Caffe等主流框架的模型格式,这大大降低了开发者的迁移成本和学习门槛。同时,TNN还提供了丰富的API和文档,以帮助开发者快速上手并进行模型推理。这些特性使得TNN成为一个非常友好、易用的移动端深度学习推理框架。

在实际应用中,TNN已经被广泛应用于各种腾讯的产品中,如微信、QQ等。无论是用于图像识别、语音识别还是自然语言处理等任务,TNN都能提供高效、准确的推理结果。同时,由于其轻量级的设计和高性能的表现,TNN还特别适合于在资源受限的环境下进行部署,如智能家居、可穿戴设备等领域。

总的来说,TNN是一个功能强大且易于使用的轻量级深度学习推理框架。它通过简洁高效的框架设计、模型优化技术和硬件加速技术等手段,实现了在移动端设备上高性能、低延迟的推理。随着移动端设备的不断普及和发展以及深度学习技术的不断进步,相信TNN将在未来的移动智能时代发挥更加重要的作用。

\subsection{MNN}

MNN(Mobile Neural Network)是阿里巴巴集团精心打造的一个轻量级深度学习推理引擎。在当前的移动计算时代,深度学习模型在移动设备上的部署与推理已成为业界关注的焦点。MNN应运而生,旨在为移动端设备提供高效、稳定的深度学习推理能力。

MNN的显著特点之一是其广泛的硬件平台和操作系统支持。无论是基于ARM架构的智能手机、平板,还是x86架构的笔记本、台式机,甚至是各种嵌入式设备和物联网设备,MNN都能轻松应对。此外,MNN还支持多种操作系统,包括Android、iOS、Linux等,为开发者提供了极大的便利性和灵活性。

除了硬件和平台的多样性支持外,MNN还提供了丰富的优化选项,以帮助开发者提高推理速度和降低资源消耗。这些优化选项包括模型压缩、量化、剪枝等,旨在减小模型的大小、降低计算复杂度和内存占用,从而实现更快的推理速度和更低的功耗。通过这些优化技术,MNN能够在保持模型精度的同时,显著提升推理性能,为移动端应用带来更好的用户体验。

在实际应用中,MNN已被广泛应用于阿里巴巴集团内部的多个业务场景中,如淘宝、天猫等电商平台的商品推荐、图像搜索等功能。同时,由于其开源的特性,MNN也吸引了众多外部开发者和研究者的关注和使用。他们基于MNN开发出了各种创新的移动应用和研究项目,进一步推动了深度学习技术在移动端的应用和发展。

总的来说,MNN是一个功能强大且易于使用的轻量级深度学习推理引擎。它通过支持多种硬件平台和操作系统、提供丰富的优化选项等手段,为开发者提供了高效、稳定的推理能力。随着移动端设备的不断普及和发展以及深度学习技术的不断进步,相信MNN将在未来的移动智能时代发挥更加重要的作用。

\subsection{MediaPipe}

MediaPipe,由谷歌公司主导开发,是一个功能强大且跨平台的多媒体处理管道构建框架。在当前的多媒体时代,从视频流、音频流到各种传感器数据,处理和分析这些多媒体数据已成为众多应用的核心需求。MediaPipe正是为了满足这些需求而诞生的。

MediaPipe的设计理念是提供一个灵活、可扩展的框架,使开发者能够轻松地构建复杂的多媒体处理管道。这些管道可以包含多种机器学习模型和计算机视觉任务,如目标检测、人脸识别、姿态估计等。为了简化开发过程,MediaPipe提供了一系列工具和库,这些工具和库封装了底层的多媒体处理细节,使开发者能够专注于应用层面的逻辑实现。

虽然MediaPipe本身不是一个纯粹的推理框架,但它可以与其他推理框架无缝结合,如TensorFlow Lite、MNN等。通过这种结合,MediaPipe能够实现多媒体数据的实时处理和分析。例如,在一个移动应用中,MediaPipe可以接收来自摄像头的视频流,使用TensorFlow Lite进行目标检测,然后将检测结果实时显示在用户界面上。

除了实时处理外,MediaPipe还支持多种应用场景。无论是虚拟现实、增强现实还是智能家居等领域,MediaPipe都能提供强大的多媒体处理能力。同时,由于其跨平台的特性,MediaPipe可以在多种操作系统和设备上运行,如Android、iOS、Linux等,这大大扩展了它的应用范围。

总的来说,MediaPipe是一个功能强大且易于使用的多媒体处理框架。它通过提供灵活的处理管道构建方式、丰富的工具和库以及与其他推理框架的结合能力,为开发者提供了强大的多媒体处理能力。随着多媒体数据的不断增长和应用需求的不断提升,相信MediaPipe将在未来的多媒体应用领域发挥更加重要的作用。

\subsection{TVM}

TVM(Tiny Versatile Machine)是一个开源的机器学习编译器栈,它的出现极大地推动了机器学习模型在多种硬件平台上的高效推理。在当前的计算环境中,从云端服务器到边缘设备,各种硬件平台对机器学习模型的推理性能要求各不相同。TVM正是为了满足这一广泛需求而设计的。

TVM的核心优势在于其对多种神经网络模型和优化策略的支持。无论是深度学习领域的经典模型,如卷积神经网络(CNN)、循环神经网络(RNN),还是新兴的模型架构,如Transformer、生成对抗网络(GAN)等,TVM都能提供高效的推理能力。此外,TVM还支持多种优化策略,包括模型压缩、量化、剪枝等,这些策略可以显著降低模型的大小和计算复杂度,从而提高推理速度和能效比。

除了对多种模型和优化策略的支持外,TVM还具备针对特定硬件平台生成优化代码的能力。这意味着开发者可以根据目标硬件的特性,如处理器架构、内存大小、功耗限制等,定制化的优化模型的推理性能。这种灵活性使得TVM能够适应各种应用场景,从高性能计算到低功耗的边缘设备推理,都能发挥出色的性能。

在实际应用中,TVM已经被广泛应用于各种领域。例如,在自动驾驶领域,TVM可以帮助车辆实时处理来自摄像头的图像数据,实现准确的物体检测和识别;在智能家居领域,TVM可以部署在智能家居设备上,实现语音识别、人脸识别等功能;在移动应用领域,TVM可以显著提升移动设备上机器学习模型的推理性能,为用户带来更好的体验。

总的来说,TVM是一个功能强大且灵活的机器学习编译器栈。它通过支持多种神经网络模型和优化策略,以及针对特定硬件平台的优化能力,为开发者提供了高效的模型推理解决方案。随着机器学习技术的不断发展和应用场景的不断拓展,相信TVM将在未来的机器学习生态中发挥更加重要的作用。

\subsection{libtorch}

libtorch,作为PyTorch的C++前端,为开发者开辟了一条在C++应用中使用PyTorch深度学习功能的途径。在深度学习领域,Python一直是最受欢迎的语言,但许多实际应用场景,特别是那些对性能有严格要求或对运行环境有限制的场景,C++往往是更合适的选择。libtorch的出现,正是为了满足这部分需求。

libtorch提供了与Python API相似的接口和功能,这意味着开发者可以在熟悉的PyTorch编程范式下,无缝地切换到C++环境。无论是模型训练、推理还是部署,libtorch都能提供强大的支持。这种一致性不仅降低了学习成本,也提高了代码的可移植性和可维护性。

尽管libtorch本身不是一个独立的推理框架,但它在PyTorch生态系统中的地位不容忽视。作为PyTorch的重要组成部分,libtorch为在C++环境中使用PyTorch模型提供了极大的便利。对于那些需要在C++应用中集成深度学习功能的开发者来说,libtorch几乎是一个不可或缺的工具。

在实际应用中,libtorch已经被广泛用于各种场景。例如,在嵌入式系统和物联网设备中,由于资源限制和性能要求,C++往往是首选的开发语言。通过libtorch,开发者可以在这些设备上轻松地部署和运行PyTorch模型,实现各种复杂的深度学习任务。

总的来说,libtorch是一个强大而灵活的工具,它将PyTorch的深度学习功能扩展到了C++领域。无论是对性能有严格要求的场景,还是需要在C++环境中集成深度学习功能的场景,libtorch都能提供出色的支持。随着深度学习技术的不断发展和应用场景的不断拓展,相信libtorch将在未来的深度学习生态中发挥更加重要的作用。

\section{推理框架的选择}

在面临多种深度学习框架的选择时,我首先考虑到了自己的开发环境。由于我个人的所有开发平台,包括我在家中的Intel Core i5-11400和在学校的Intel Xeon E5 2690 v4处理器,都是基于Intel架构的,这使得OpenVINO平台成为了一个非常合适的选择。OpenVINO是Intel推出的针对自家硬件优化的开源计算机视觉库,它能够充分利用Intel硬件的性能优势,提供高效的推理能力。

因此,我首先开发了一个基于Python和OpenVINO的版本,这个版本能够在我的酷睿以及至强平台进行测试。然而,我并不仅仅满足于在个人电脑上运行我的程序,我还希望能够让老师和其他同学也能方便地测试和使用我的程序。

考虑到Windows客户端可能存在的兼容性问题,以及安装运行库可能带来的复杂性,我决定开发一个更加通用和易用的Android客户端。这样一来,无论是使用Windows、macOS还是Linux操作系统的用户,只要他们的设备上安装了Android模拟器或者拥有Android设备,就能够轻松地运行和测试我的程序。

为了实现这一目标,我选择了支持全平台的NCNN框架。NCNN是一个为移动端设计的高效神经网络前向计算框架,它支持多种操作系统和硬件平台,非常适合用于开发跨平台的深度学习应用。我利用NCNN的C++ API,成功地开发了Android版本、VC++版本以及C\#.Net版本的程序,这些版本都能够在不同平台上顺利运行,并且达到了预期的效果。

\section{模型的验证}


	\chapter{模型量化的探究}
\label{chapter:6}

\section{介绍}

NVIDIA于2018年推出了基于Turing架构的革新性GPU,其中最引人注目的亮点莫过于全新的Tensor Core技术。这一技术利用了专门的ASIC(应用特定集成电路)来加速FP16(半精度浮点数)矩阵运算,为深度学习领域带来了显著的性能提升。FP16的位宽仅为FP32(单精度浮点数)的一半,虽然在一定程度上牺牲了数值精度,但实际应用中模型的准确度并未受到严重影响。因此,通过采用半精度量化技术,不仅可以大幅提升模型的推理速度,还能有效降低模型所需的存储空间。更为灵活的是,Turing架构还支持混合精度训练,允许在某些层使用半精度进行加速,同时保持其他层的单精度以确保准确性。

然而,我当前使用的是基于Pascal架构的NVIDIA Tesla P40 GPU。与Turing架构相比,Pascal架构并未配备专门的Tensor Core单元,因此在加速半精度计算方面存在一定局限。正因如此,在我的深度学习训练中,我主要依赖于纯单精度(FP32)进行计算。尽管如此,Pascal架构依然以其卓越的性能和稳定性在深度学习领域占有一席之地,为众多研究者和开发者提供了强大的计算支持。

\section{数据类型介绍}

\subsection{FP32}

FP32,即单精度浮点数(Single-Precision Floating-Point),是一种计算机中用于表示浮点数的数据类型。FP32使用32位(即4字节)来存储一个浮点数,其中通常包括1位符号位、8位指数位和23位尾数位(有效数字位)。

FP32的特点如下:

\begin{enumerate}
	\item 动态范围:由于有8位用于指数,FP32可以表示非常大或非常小的数,动态范围较广。
	\item 精度:23位的尾数提供了相对较高的精度,适用于需要精确计算的科学和工程应用。
	\item 通用性:FP32是大多数计算机系统和编程语言的默认浮点数表示,具有广泛的软件和硬件支持。
	\item 性能:虽然在现代硬件上,针对半精度(FP16)或混合精度(如TF32、BF16等)的优化可能提供更高的性能,但FP32仍然是许多高性能计算(HPC)和深度学习应用中不可或缺的格式,特别是在需要高精度的场景下。
	\item 稳定性:在某些复杂的数值计算中,使用FP32可以减少数值不稳定性和舍入误差。
\end{enumerate}

在深度学习中,FP32一度是训练和部署模型的标准格式,因为它提供了足够的精度来确保模型的正确性和稳定性。然而,随着模型规模的增大和对计算效率要求的提高,人们开始探索使用更低精度的格式(如FP16、BF16、INT8等)来加速训练和推理过程,同时尽量保持模型的性能。尽管有这些低精度格式的挑战者,FP32仍然在许多需要高精度或复杂数值处理的深度学习应用中占据重要地位。

\subsection{FP64}

FP64,即双精度浮点数(Double-Precision Floating-Point),是一种计算机中用于表示浮点数的数据类型,它使用64位(即8字节)来存储一个浮点数。这64位通常被分配为1位符号位、11位指数位和52位尾数位(有效数字位)。

FP64的特点如下:

\begin{enumerate}
	\item 高精度:由于有52位的尾数,FP64提供了非常高的数值精度,适用于需要极高精确度的科学和工程计算,如天文学、物理学模拟、金融建模等。
	\item 大动态范围:11位的指数位允许FP64表示极大和极小的数值,这使得它能够处理广泛的数值范围,而不会遇到溢出或下溢的问题。
	\item 稳定性:在涉及大量计算和复杂数学运算的应用中,使用FP64可以减少舍入误差和数值不稳定性,从而得到更可靠和准确的结果。
	\item 广泛的软件支持:几乎所有的编程语言和计算系统都支持FP64数据类型,这保证了其在各种应用中的通用性和兼容性。
	\item 性能考量:尽管FP64提供了高精度和高稳定性,但它也带来了更高的计算和存储成本。相比于FP32或更低精度的格式,FP64的计算速度通常较慢,且占用的内存和存储空间更大。因此,在性能敏感的应用中,如实时系统、嵌入式设备或大规模数据分析中,可能会优先考虑使用更低精度的数据类型。
\end{enumerate}

在深度学习中,FP64通常不是首选的数据类型,因为深度学习模型往往可以容忍一定程度的数值误差,并且更关注于计算效率和内存使用。然而,在某些特定场景下,如需要极高精度的模型训练、复杂的数值分析或与其他高精度科学计算软件的交互中,FP64可能是必要的。

\subsection{FP16}

FP16量化,即半精度浮点数量化,是一种在深度学习和机器学习领域中常用的优化技术。FP16指的是使用16位(bit)来表示浮点数,相较于传统的32位浮点数(FP32,即单精度浮点数),它减少了一半的存储空间和内存带宽需求。

在深度学习模型中,权重、激活值和梯度等参数通常都是以浮点数形式存储和计算的。传统的FP32提供了较高的数值精度和动态范围,但在很多深度学习应用中,并不需要这么高的精度。实际上,使用FP16往往可以在保证模型准确性的同时,显著提升计算性能和存储效率。

FP16量化的主要优势包括:

\begin{enumerate}
	\item 性能提升:半精度计算可以显著减少内存访问和传输的开销,因为数据的大小减少了一半。这对于需要大量数据传输的GPU加速计算尤为重要。
	\item 存储节省:模型的大小减半,这对于部署到存储空间有限的设备上(如移动设备、嵌入式系统)非常有利。
	\item 功耗降低:更少的内存访问和数据传输意味着更低的功耗,这对于电池供电的设备尤为重要。
	\item 硬件加速:一些现代GPU和AI加速器提供了对FP16计算的专门优化,如NVIDIA的Tensor Cores,这些硬件可以显著加速半精度计算。
\end{enumerate}

然而,使用FP16量化也需要注意一些问题:

\begin{enumerate}
	\item 数值稳定性:由于FP16的精度较低,可能会导致梯度消失或爆炸,从而影响模型的训练稳定性。因此,可能需要使用混合精度训练(Mixed Precision Training)等技术来平衡性能和稳定性。
	\item 动态范围:FP16的动态范围比FP32小,可能无法表示非常大或非常小的数值。这可能需要通过适当的缩放或归一化技术来处理。
	\item 软件支持:虽然许多深度学习框架和库都支持FP16计算,但仍然需要确保使用的软件和工具链完全兼容并能够正确处理半精度数值。
\end{enumerate}

总的来说,FP16量化是一种有效的深度学习优化技术,可以在保证模型性能的同时显著提高计算效率和存储效率。

\subsection{INT8}

INT8量化是一种模型优化技术,旨在将深度学习模型中的权重和激活值从原始的32位浮点数(FP32)格式转换为8位整数(INT8)格式,以减小模型大小、降低功耗并加快计算速度。

INT8量化的主要优势包括:
\begin{enumerate}
	\item 减小模型大小:通过使用8位整数代替32位浮点数,可以显著减小模型的大小,这对于部署到资源受限的设备(如移动设备或嵌入式系统)上非常有利。
	\item 加快计算速度:在许多硬件平台上,整数运算比浮点运算更快,因此INT8量化可以加速模型的推理速度。
	\item 降低功耗:整数运算通常比浮点运算更节能,这对于电池供电的设备尤为重要。
	然而,INT8量化也面临一些挑战:
\end{enumerate}

\begin{enumerate}
	\item 精度损失:由于从FP32转换到INT8会损失一些精度,这可能会影响模型的性能。因此,在量化过程中需要采取一些策略来最小化精度损失,例如使用校准数据集来调整量化参数。
	\item 硬件和软件支持:为了充分利用INT8量化的优势,需要硬件和软件的支持。一些现代的处理器和加速器提供了对INT8运算的优化,而深度学习框架和库也需要提供对INT8量化的支持。
\end{enumerate}

INT8量化的过程通常包括以下几个步骤:

\begin{enumerate}
	\item 模型训练:首先,使用FP32格式训练深度学习模型。
	\item 校准数据集:选择一个代表性的数据集作为校准数据集,用于调整量化参数。
	\item 量化参数计算:使用校准数据集计算量化参数(如缩放因子和偏移量),这些参数将用于将FP32值映射到INT8范围。
	\item 模型量化:使用计算得到的量化参数将模型的权重和激活值从FP32转换为INT8。
	\item 模型验证:验证量化后的模型在性能上是否与原始模型相似。如果需要,可以对量化后的模型进行微调以恢复性能。
\end{enumerate}

总的来说,INT8量化是一种有效的深度学习模型优化技术,可以显著减小模型大小、加快计算速度并降低功耗。然而,为了充分利用其优势,需要仔细选择校准数据集、计算量化参数,并确保硬件和软件的支持。

\subsection{BF32}

BF32(Bfloat16) 是一种浮点数表示格式,它使用16位(2字节)来存储数据,但具有与常规的16位浮点数(即FP16)不同的位布局和指数范围。BFloat16格式特别设计用于深度学习和其他需要高精度但不需要全范围32位浮点数的应用。

在BFloat16格式中,1位用于符号(s),8位用于指数(e),而7位用于尾数(m,也称为有效数字或分数)。这种布局提供了比标准FP16更大的指数范围,但牺牲了尾数的精度。这种权衡使得BFloat16在处理深度学习中的大范围数值时更为有效,同时减少了存储和计算需求。

BF32的优势主要包括:

\begin{enumerate}
	\item 减少存储需求:与FP32相比,BF32将存储需求减半,这对于大规模模型和数据集来说非常重要。
	\item 加速计算:许多现代硬件平台(如GPU和TPU)都针对BF32优化,从而可以加速深度学习训练和推理。
	\item 降低功耗:减少存储和计算需求通常也意味着更低的功耗,这对于数据中心和边缘设备都是重要的考虑因素。
\end{enumerate}

保持足够的精度:尽管BF32的尾数精度较低,但在许多深度学习场景中,它仍能提供足够的精度来维持模型性能。

需要注意的是,虽然BF32在深度学习领域很有用,但它并不总是适用于所有类型的计算或所有模型。在某些情况下,使用FP32或FP16可能更为合适,具体取决于模型的复杂性、所需的精度以及硬件支持等因素。因此,在选择数值表示格式时,需要根据具体的应用场景和需求进行权衡。

\subsection{TF32}

TF32(TensorFloat-32) 是 NVIDIA 为其 Ampere 架构的 GPU 引入的一种新的数学模式,专为深度学习而设计。TF32 旨在结合 FP16(半精度浮点数)和 FP32(单精度浮点数)的优势,以在深度学习训练和推理中实现更高的吞吐量和更快的速度,同时保持与 FP32 相当的精度。

TF32 的工作原理是在内部使用 19 位尾数(mantissa)和 10 位指数(exponent)来表示浮点数,这与标准的 FP32 和 FP16 格式不同。这种表示法允许 TF32 在处理深度学习中的大范围数值时保持较高的精度,同时减少了存储和计算需求。然而,需要注意的是,TF32 并不是一种标准的 IEEE 浮点数格式,而是 NVIDIA 专为其 GPU 平台定制的一种格式。

使用 TF32 的主要优势包括:

\begin{enumerate}
	\item 性能提升:TF32 模式允许 GPU 在执行深度学习操作时实现更高的吞吐量和更快的速度。这主要归功于 TF32 在内部处理数据时所使用的优化算法和硬件加速。
	\item 精度保持:尽管 TF32 使用了较少的位数来表示浮点数,但 NVIDIA 声称它在许多深度学习应用中能够提供与 FP32 相当的精度。这使得研究人员和开发人员能够在不牺牲模型性能的情况下加快训练和推理速度。
	\item 硬件支持:TF32 是 NVIDIA Ampere 架构 GPU 的一项特性,因此只有支持这一架构的 GPU 才能利用 TF32 的优势。对于拥有兼容硬件的用户来说,使用 TF32 可以是一种无需修改代码即可提升性能的有效方法。
	\item 软件兼容性:NVIDIA 的深度学习框架和库(如 TensorFlow、PyTorch 等)通常都支持 TF32。这意味着开发人员可以在不修改现有代码的情况下利用 TF32 的优势,只需确保他们的 GPU 和软件环境支持这一特性即可。
\end{enumerate}

然而,需要注意的是,尽管 TF32 在许多情况下都能提供显著的性能提升和精度保持,但它并不总是适用于所有类型的深度学习模型和应用。在某些特定场景下,使用 FP32 或其他数值表示格式可能更为合适。因此,在选择使用哪种数值表示格式时,需要根据具体的应用场景和需求进行权衡和测试。

\section{ONNX量化步骤}

\subsection{半精度FP16量化}

具体文件请参考\url{Train/SHMTU_CAS_OCR_RESNET/src/classify/digit/quantize_onnx_fp16.py}。

\begin{lstlisting}[caption={ONNX FP16量化},language=Python,label=code:onnx_fp16]
	import warnings

	import onnx
	from onnxconverter_common import float16

	model = onnx.load("resnet34_digit_latest.onnx")

	with warnings.catch_warnings():
	warnings.filterwarnings(
	"ignore", category=UserWarning,
	message="the float32 number .* will be truncated to .*"
	)
	model_fp16 = float16.convert_float_to_float16(model)

	onnx.save(model_fp16, "resnet34_digit_latest_fp16.onnx")

\end{lstlisting}

该Python代码片段(\ref{code:onnx_fp16})执行了以下步骤:

\begin{enumerate}
	\item {
		导入必要的库和模块:

		\begin{itemize}
			\item \texttt{warnings}:用于处理警告信息。
			\item \texttt{onnx}:用于加载和保存ONNX模型。
			\item \texttt{onnxconverter\_common.float16}:提供了将ONNX模型从float32转换为float16的功能。
		\end{itemize}
	}

	\item {加载模型

		加载预训练的ONNX模型("resnet34\_digit\_latest.onnx")到变量\texttt{model}。
	}

	\item {
		使用\texttt{warnings.catch\_warnings()}上下文管理器来忽略特定类型的警告信息。

		通过\texttt{warnings.filterwarnings()}函数,忽略了与float32到float16转换相关的用户警告。
	}

	\item {
		模型量化

		调用\texttt{float16.convert\_float\_to\_float16()}函数将\texttt{model}从float32格式转换为float16格式,并将转换后的模型存储在变量\texttt{model\_fp16}中。
	}

	\item {
		使用\texttt{onnx.save()}函数保存
	}

\end{enumerate}

此代码的目的是减小模型文件的大小并可能提高推理速度,但可能会以牺牲一些模型精度为代价。

\subsection{INT8动态量化}

具体文件请参考\url{Train/SHMTU_CAS_OCR_RESNET/src/classify/digit/quantize_onnx_int8.py}。

\begin{lstlisting}[caption={ONNX INT8量化},language=Python,label=code:onnx_int8]
	import onnx
	from onnxruntime.quantization import preprocess
	from onnxruntime.quantization import quantize_dynamic, QuantType

	from pathlib import Path

	model_fp32 = "resnet34_digit_latest_p.onnx"
	model_quant = "resnet34_digit_latest_int8.onnx"

	model_fp32 = Path(model_fp32)
	model_quant = Path(model_quant)

	# python -m onnxruntime.quantization.preprocess --input resnet34_digit_latest.onnx --output resnet34_digit_latest_p.onnx

	quantize_dynamic(
	model_fp32,
	model_quant,
	weight_type=QuantType.QUInt8,
	)

\end{lstlisting}

该Python代码片段(\ref{code:onnx_int8})执行了以下步骤:

\begin{enumerate}
	\item 导入必要的库:
	\begin{enumerate}
		\item \texttt{onnx}: 用于加载和保存ONNX模型。
		\item \texttt{onnxruntime.quantization.preprocess}: 用于预处理模型,准备进行量化。
		\item \texttt{onnxruntime.quantization.quantize\_dynamic} 和 \texttt{QuantType}: 用于执行动态量化。
		\item \texttt{pathlib.Path}: 用于处理文件路径。
	\end{enumerate}
	\item 定义输入和输出模型的路径:
	\begin{enumerate}
		\item \texttt{model\_fp32}: 指向原始的32位浮点数模型文件。
		\item \texttt{model\_quant}: 指向将要保存的量化后的8位整数模型文件。
	\end{enumerate}
	\item 使用 \texttt{Path} 类将字符串路径转换为 \texttt{Path} 对象,以便进行更方便的路径操作。
	\item 注释行(以`\#`开头)指示了如何使用命令行工具对模型进行FP16量化处理,这是官方文档中推荐的做法,首先进行FP16量化后再进行INT8量化。
	\item 调用 \texttt{quantize\_dynamic} 函数对模型进行动态量化:
	\begin{enumerate}
		\item 输入模型:\texttt{model\_fp32}
		\item 输出模型:\texttt{model\_quant}
		\item 权重类型:\texttt{QuantType.QUInt8},表示使用无符号8位整数进行量化。
	\end{enumerate}
\end{enumerate}

量化后的模型可用于减少模型大小和提高推理速度,同时保持模型的准确性。这对于在资源受限的设备上部署模型特别有用。

这里使用的是动态INT8的方法进行模型的量化,下面在NCNN中将介绍静态量化的方法。

\section{NCNN量化步骤}

我们使用官方提供的转换工具进行操作。

由于ONNX使用Google Protobuf进行序列化,然而Linux操作系统中的Protobuf或通过包管理器(如APT)安装的Protobuf有很大概率与官方Github Release提供的预编译版本所需的Protobuf版本不同,因此我们选择兼容性极好的Windows来执行NCNN的量化操作。

\subsection{半精度FP16量化}

NCNN FP16量化脚本(\ref{code:ncnn_fp16})请参考\url{Train/SHMTU_CAS_OCR_RESNET/convert_onnx_to_ncnn.ps1}。

\begin{lstlisting}[caption={NCNN FP16量化},label=code:ncnn_fp16]
	# Convert ONNX model to NCNN format
	# Usage: powershell -File convert_onnx_to_ncnn.ps1

	function ConvertModelToNCNN {
		param (
		[string]$modelName
		)

		$toolPath = ".\3rdparty\ncnn\bin"
		$pthDirPath = ".\workdir\Models"

		Write-Host "Converting $modelName to NCNN format"

		& "$toolPath\onnx2ncnn.exe" `
		"$pthDirPath\$modelName.onnx" `
		"$pthDirPath\$modelName.fp32.param" `
		"$pthDirPath\$modelName.fp32.bin"

		# Optimize quantization, 1: fp16, 0: fp32
		& "$toolPath\ncnnoptimize.exe" `
		"$pthDirPath\$modelName.fp32.param" `
		"$pthDirPath\$modelName.fp32.bin" `
		"$pthDirPath\$modelName.fp16.param" `
		"$pthDirPath\$modelName.fp16.bin" `
		1
	}

	ConvertModelToNCNN -modelName "resnet18_equal_symbol_latest"
	ConvertModelToNCNN -modelName "resnet18_operator_latest"
	ConvertModelToNCNN -modelName "resnet34_digit_latest"

\end{lstlisting}

ncnnoptimize工具能够自动将FP32模型量化为FP16模型,而且精度损失不会太大。

\subsection{INT8静态量化}

\begin{enumerate}
	\item 将ONNX转换为NCNN格式
	\item 进行FP16量化(可选)
	\item 准备一组数据用于校准
	\item 使用ncnn2table生成校准表
	\item 使用ncnn2int8进行INT8量化
\end{enumerate}

\subsubsection{生成校准表}

\begin{lstlisting}[caption={生成校准表},label=code:ncnn_int8_table]
	./ncnn2table `
		r34_digit.param `
		r34_digit.bin image_digit_list.txt `
		r34_digit.table `
		mean=[123.675,116.280,103.530] `
		norm=[0.017,0.018,0.017] `
		shape=[1,3,224,224] `
		pixel=RGB `
		thread=8 `
		method=kl
\end{lstlisting}

具体命令行请参考代码\ref{code:ncnn_int8_table}

\begin{enumerate}
	\item 使用"find image\_digit/ -type f > image\_digit\_list.txt"将图片文件名生成txt格式列表
	\item 使用ncnn2table工具生成校准表
\end{enumerate}

\subsubsection{静态量化}

将FP32模型或FP16模型,以及校准表传入ncnn2int8工具即可进行INT8量化,具体命令行请参考代码\ref{code:ncnn_int8_q}。

\begin{lstlisting}[caption={NCNN INT8量化},label=code:ncnn_int8_q]
	./ncnn2int8 r34_digit.param r34_digit.bin r34_digit_8.param r34_digit_8.bin r34_digit.table
\end{lstlisting}

\section{实验}

\subsection{FP16}

经过Selenium的验证,FP16模型能够有90\%的正确率,因此适合进行部署。

\subsection{INT8}

经过试验,INT8量化阿拉伯数字识别模型后无法正确识别数字3,因此不适合使用INT8量化。

	\chapter{Python 推理}
\label{chapter:7}

在Python中进行深度学习模型的推理是一个重要的环节,它涉及到将训练好的模型应用于实际数据以获取预测结果。本章节将详细介绍使用不同工具和框架进行模型推理的方法。

\section{PyTorch直接推理}

PyTorch作为一个流行的深度学习框架,提供了方便的工具来进行模型的推理。通过调用model.eval()方法,我们可以将模型设置为评估模式,此时模型的参数将不再更新,适用于进行推理测试。在评估模式下,我们可以利用训练好的模型对新的数据进行预测,并获取相应的输出结果。

具体代码请参考\url{Train/SHMTU_CAS_OCR_RESNET/src/classify/predict/predict_file.py}。

\section{ONNX Runtime}

ONNX(Open Neural Network Exchange)是一种开放的深度学习模型表示格式,得到了众多深度学习框架的支持。通过将模型导出为ONNX格式,我们可以利用ONNX Runtime在不同平台和设备上进行模型的推理。ONNX Runtime是一个高效的推理引擎,支持多种硬件加速,包括CPU、GPU和专用加速器等。通过使用ONNX Runtime,我们可以方便地在不同环境中部署和推理模型。

具体的ONNX Runtime推理代码可以参考位于\url{Train/SHMTU_CAS_OCR_RESNET/src/classify/digit/infer_onnx.py}路径下的代码文件。

\section{Intel OpenVINO}

考虑到大部分PC和服务器使用的是Intel处理器,而且Intel还提供了神经计算棒等外接设备,可以方便地扩展嵌入式设备如树莓派的深度网络推理能力,我们选择使用Intel OpenVINO工具套件进行推理。OpenVINO是Intel推出的一个优化推理性能的开源工具,可以充分利用Intel硬件的优势,提供高效的推理性能。

在市场上,二手Intel神经计算棒的价格相对较低,这使得开发者可以轻松地获取到这些设备进行开发和测试。图\ref{fig:intelstick}展示了Intel神经计算棒的外观。

\begin{figure}
	\centering
	\includegraphics[width=0.5\linewidth]{Resources/Picture/intel_stick}
	\caption{Intel神经计算棒}
	\label{fig:intelstick}
\end{figure}

具体使用Intel OpenVINO进行推理的代码可以参考\url{Train/SHMTU_CAS_OCR_RESNET/src/classify/digit/openvino_test.py}路径下的代码文件。通过这些代码,我们可以了解如何加载模型、设置输入数据以及执行推理等关键步骤。

我们选择Python进行原型设计,但是OpenVINO也提供了C++ API,在将来我们也可以很方便地将程序改写为C++以提升运行效率。

	\chapter{Microsoft Visual C++的部署}
\label{chapter:8}

\section{NCNN API}

\subsection{工作流程}

ncnn是一个为移动端设计的高效神经网络前向计算框架。以下是使用ncnn的C++ API进行神经网络推断的基本步骤:

\begin{enumerate}

	\item \textbf{包含必要的头文件}

	在使用ncnn之前,需要包含必要的头文件。这通常包括ncnn的主要命名空间和一些辅助工具。

	\begin{lstlisting}[caption={导入OpenCV与NCNN},language=C++]
		#include <net.h>
		#include <opencv2/opencv.hpp>
	\end{lstlisting}

	\item \textbf{加载预训练模型和参数}

	ncnn可以从硬盘加载已经转换好的预训练模型,这些模型通常由其他框架(如Caffe、TensorFlow等)转换而来。

	\begin{lstlisting}[caption={加载权重文件},language=C++]
		ncnn::Net net;
		net.load_param("model.param");
		net.load_model("model.bin");
	\end{lstlisting}

	\item \textbf{预处理输入数据}

	根据模型的输入要求,对输入数据进行预处理。这可能包括缩放、裁剪、归一化等操作。

	\begin{lstlisting}[caption={读取图片并处理数据},language=C++]
		cv::Mat img = cv::imread("input.jpg");
		ncnn::Mat in = ncnn::Mat::from_pixels_resize(img.data, ncnn::Mat::PIXEL_BGR, img.cols, img.rows, model_input_width, model_input_height);
		in.substract_mean_normalize(mean_vals, norm_vals);
	\end{lstlisting}

	请注意,`mean\_vals` 和 `norm\_vals` 需要根据你的模型设置正确的均值和标准差。

	\item \textbf{创建提取器并设置输入}

	创建一个提取器对象,并将预处理后的数据设置为输入。

	\begin{lstlisting}[caption={设置输入输出},language=C++]
		ncnn::Extractor ex = net.create_extractor();
		ex.input("input_blob", in);
	\end{lstlisting}

	这里的`"input\_blob"`是模型中定义的输入层的名称,需要根据实际情况进行替换。

	\item \textbf{执行推断}

	调用提取器的`extract`方法来执行推断。

	\begin{lstlisting}[caption={推理},language=C++]
		ncnn::Mat out;
		ex.extract("output_blob", out);
	\end{lstlisting}

	这里的`"output\_blob"`是模型中定义的输出层的名称,需要根据实际情况进行替换。

	\item \textbf{后处理输出数据}

	根据需要对输出数据进行后处理,例如将结果解码为可读的格式或应用于其他任务。

\end{enumerate}

这些步骤提供了使用ncnn进行神经网络推断的基本框架。具体细节可能因模型和数据而异,需要根据具体情况进行调整。

\subsection{NCNN调用Vulkan进行GPU加速}

ncnn是一个高效的神经网络前向计算框架,支持使用Vulkan进行加速。Vulkan是一个跨平台的计算机图形和计算API,通常用于游戏和图形应用程序,但也可以用于加速神经网络的推理。

使用Vulkan加速ncnn的一般步骤如下:

\begin{enumerate}
	\item 安装Vulkan:首先,你需要在你的系统上安装Vulkan。这通常涉及到下载和安装Vulkan SDK,并配置相关的环境变量。具体的安装步骤可能因操作系统和硬件的不同而有所不同。
	\item 编译ncnn with Vulkan:在安装了Vulkan之后,你需要使用Vulkan来编译ncnn。这通常涉及到在编译ncnn时启用Vulkan支持。具体的编译步骤可能因ncnn的版本和你的开发环境而有所不同。
	\item 使用编译好的ncnn进行推理:一旦你使用Vulkan编译了ncnn,你就可以使用编译好的ncnn来进行神经网络的推理了。在推理时,ncnn会自动使用Vulkan进行加速。
\end{enumerate}

需要注意的是,使用Vulkan加速ncnn并不总是能提供显著的性能提升。这取决于多种因素,包括你的硬件、操作系统、ncnn的版本以及你正在运行的神经网络模型。在某些情况下,使用Vulkan可能会降低性能,因此在使用之前最好进行充分的测试和性能分析。

另外,ncnn还支持其他加速方式,如使用ARM NEON指令集进行加速等。你可以根据你的硬件和需求选择最适合的加速方式。

\begin{lstlisting}[caption={NCNN with Vulkan Initialization}, language=C++]
	#include "net.h"
	// ... 其他必要的NCNN和Vulkan头文件 ...

	int main() {
		// 初始化Vulkan设备和实例(伪代码)
		// vulkanDevice = initializeVulkanDevice(...);
		// vulkanQueueFamilyIndex = getVulkanQueueFamilyIndex(...);

		// 初始化NCNN网络,并启用Vulkan支持
		ncnn::Net net;
		net.opt.use_vulkan_compute = true;  // 启用Vulkan计算
		// 设置Vulkan设备和队列家族索引(如果有必要的话)
		// net.set_vulkan_device(vulkanDevice);
		// net.set_vulkan_queue_family_index(
		//     vulkanQueueFamilyIndex
		// );

		// 加载模型
		net.load_param("model.param");
		net.load_model("model.bin");

		// 准备输入数据(伪代码)
		// ncnn::Mat input = prepareInputData(...);

		// 运行网络进行推理
		// ncnn::Extractor extractor = net.create_extractor();
		// extractor.input("input_blob", input);
		// extractor.extract("output_blob", output);

		// 处理输出数据(伪代码)
		// processOutputData(output);

		return 0;
	}
\end{lstlisting}

\section{Visual Studio 解决方案}

\begin{figure}
	\centering
	\includegraphics[width=0.7\linewidth]{Resources/Picture/vs}
	\caption{Visual Studio解决方案Project}
	\label{fig:vs}
\end{figure}

\begin{enumerate}
	\item SHMTU\_CAS\_OCR\_Demo\_VC
	\item NCNN\_CLI
	\item SHMTU\_CAS\_CLR\_ClassLibrary
	\item SHMTU\_CAS\_OCR\_CLR\_Connector\_ClassLibrary
	\item SHMTU\_CAS\_OCR\_Demo\_WPF
\end{enumerate}

\subsection{SHMTU\_CAS\_OCR\_Demo\_VC}

\subsection{NCNN\_CLI}

\subsection{SHMTU\_CAS\_CLR\_ClassLibrary}

\subsection{SHMTU\_CAS\_OCR\_CLR\_Connector\_ClassLibrary}

\subsection{SHMTU\_CAS\_OCR\_Demo\_WPF}

\section{第三方库}

\subsection{NCNN20240102}

Github上提供的预先构建的NCNN Release版本采用了Release构建类型,这种类型主要优化的是程序的运行速度,同时会去除调试信息以减小文件大小。然而,这样的构建版本并不适合在Visual Studio(VC)中进行调试,因为它不包含必要的调试符号和信息。为了能够在VC中顺利进行NCNN的调试工作,我决定直接从Github仓库中克隆原始的NCNN代码,并在本地使用适当的配置重新编译它,以确保生成的库文件包含完整的调试信息。

\subsection{Protobuf 3.11.2}

ONNX(Open Neural Network Exchange)是一种用于表示深度学习模型的开放标准,它使用Google的Protobuf(Protocol Buffers)来进行模型数据的序列化和反序列化。因此,要在使用NCNN处理ONNX模型时,Protobuf是一个必不可少的依赖项。根据NCNN的官方文档推荐,Protobuf 3.11.2版本与NCNN的兼容性较好。基于这一信息,我选择了这个版本进行编译,以确保后续工作的顺利进行。

\subsection{OpenCV 5.0.0}

在已经决定自己编译Protobuf和NCNN的情况下,我考虑到OpenCV(Open Source Computer Vision Library)也是一个经常与NCNN一起使用的库,而且它的最新版本可能会带来一些新的功能和性能提升。虽然当前的稳定版本是4.9.0,但OpenCV的5.x版本已经在开发过程中,并且其源代码已经在Github上公开。因此,我决定直接从Github上克隆OpenCV的5.x开发分支,并在本地进行编译,以生成最新的5.0.0版本。这样一来,我就可以在我的项目中使用到最新版本的OpenCV,同时确保它与我自己编译的NCNN和Protobuf库之间的兼容性。

\section{编译步骤}

\subsection{安装Visual Studio 2022}

\begin{enumerate}
	\item 前往Visual Studio官网(\url{https://visualstudio.microsoft.com/})下载Visual Studio 2022安装程序。
	\item 运行安装程序,并根据提示选择需要的组件,例如Visual C++等。
	\item 等待安装完成,确保安装过程中没有错误。
\end{enumerate}

\subsection{安装CMake}

\begin{enumerate}
	\item 前往CMake官网(\url{https://cmake.org/})下载适用于Windows的CMake安装程序。
	\item 运行安装程序,并按照默认设置进行安装。
	\item 将CMake添加到系统路径中,以便在命令行中使用。
\end{enumerate}

\subsection{编译Protobuf}

\subsubsection{一般步骤}

\begin{enumerate}
	\item 前往Protobuf的GitHub仓库(\url{https://github.com/protocolbuffers/protobuf})克隆或下载源代码。
	\item 如果下载的是源代码压缩包,请解压到一个合适的目录。
	\item 打开命令行工具,进入Protobuf源代码目录。
	\item 使用CMake生成构建文件,例如运行命令:\texttt{cmake -G "Visual Studio 17 2022" ..}(根据实际情况调整生成器)。
	\item 使用Visual Studio或MSBuild编译生成的解决方案或项目文件。
	\item 编译完成后,将生成的库和头文件放置到适当的位置,以便后续使用。
\end{enumerate}

\subsection{安装Vulkan SDK}

\begin{enumerate}
	\item 前往Vulkan SDK的下载页面(\url{https://vulkan.lunarg.com/sdk/home})下载适用于Windows的最新版Vulkan SDK安装程序。
	\item 运行安装程序,并按照提示进行安装。通常情况下,安装程序会自动配置环境变量和路径。
	\item 验证安装是否成功,可以通过检查Vulkan SDK的安装目录和相关的环境变量来确认。
\end{enumerate}

\subsection{编译NCNN}

\begin{enumerate}
	\item 前往NCNN的GitHub仓库(\url{https://github.com/Tencent/ncnn})克隆源代码到本地计算机上。
	\item 打开命令行工具,进入NCNN源代码目录。
	\item 使用CMake生成构建文件,指定需要的编译器和其他配置选项。例如,运行命令:\texttt{cmake -DCMAKE\_BUILD\_TYPE=Release ..}。
	\item 使用Visual Studio或MSBuild编译生成的解决方案或项目文件。确保选择了正确的配置(例如Release)和平台(例如x64)。
	\item 编译完成后,将生成的库和头文件放置到适当的位置,以便在项目中使用。
\end{enumerate}

\subsection{编译OpenCV}

\begin{enumerate}
	\item 前往OpenCV的GitHub仓库(\url{https://github.com/opencv/opencv})和opencv\_contrib仓库(\url{https://github.com/opencv/opencv_contrib})克隆源代码。
	\item 创建一个构建目录,并在命令行工具中进入该目录。
	\item 使用CMake配置OpenCV的构建,指定源代码路径、构建路径和opencv\_contrib模块的路径。
	\item 根据需要配置其他CMake选项,例如启用或禁用特定的功能模块、指定编译器等。
	\item 使用Visual Studio或MSBuild编译生成的解决方案或项目文件。这可能需要一些时间,具体取决于您的计算机性能和配置选项。
	\item 编译完成后,您可以选择安装OpenCV库和头文件到系统目录,或者将它们复制到您的项目目录中以便使用。确保在您的项目中正确配置库路径和头文件路径。
\end{enumerate}

\section{CAS\_OCR.cpp}

\section{CAS\_OCR API的使用}

\section{运行截图}

\begin{figure}
	\centering
	\includegraphics[width=0.7\linewidth]{Resources/Picture/vc_main}
	\caption{VC WinMain}
	\label{fig:vcmain}
\end{figure}

\begin{figure}
	\centering
	\includegraphics[width=0.7\linewidth]{Resources/Picture/vc_main_memory}
	\caption{VC GUI 内存使用(加载模型后)}
	\label{fig:vcmainmemory}
\end{figure}

\section{内存占用分析}



	\chapter{Microsoft Windows .Net 平台的部署及混合编译的探究}
\label{chapter:9dotnet}

\section{背景}

\section{解决方案}

\subsection{Microsoft ML Nuget包}

没有opencv

\subsection{通过CLR调用C++}

\section{工作流程}

\begin{enumerate}
	\item .Net WPF显示界面
	\item C\#下载图片或从本地加载图片
	\item C\#程序通过CLR Connector这个C\#的中间层与CLR进行连接
	\item CLR C++为C\#提供接口
	\item CAS\_OCR.cpp相关逻辑
\end{enumerate}

\section{CLR程序设计}

\subsection{CLR静态库}

我们无法将CLR的动态库编译为静态库,Visual C++编译器报错提示"/clr"与"/mt"两个参数不可以同时使用,而后者为编译动态库的参数,因此只能使用"/MDd"参数编译动态链接库。因此.Net C\#客户端并没有静态库版本。

\section{CLR Connector程序设计}

\section{运行截图}

\begin{figure}
	\centering
	\includegraphics[width=0.7\linewidth]{Resources/Picture/wpf_main}
	\caption{WPF 主界面}
	\label{fig:wpfmain}
\end{figure}

\begin{figure}
	\centering
	\includegraphics[width=0.7\linewidth]{Resources/Picture/wpf_memory}
	\caption{WPF程序内存占用(加载模型后)}
	\label{fig:wpfmemory}
\end{figure}

	\newpage
\chapter{Android平台的部署及C++混合编译的探究}
\label{chapter:10}

\section{程序介绍}

上海海事大学验证码识别Demo是一个结合了多种技术的综合性项目。其核心代码基于第\ref{chapter:8}章的VC++代码中的"CAS\_OCR.cpp",经过精心移植和优化,成功地在Android平台上实现了功能。为了实现跨语言调用,该项目巧妙地运用了JNI(Java Native Interface)技术,将C++代码与Java代码混合编译,使得Java能够直接调用C++中的核心功能。

\section{开发环境概述}

为了确保项目的顺利进行,我们选择了适当的开发环境和工具。项目中使用的Kotlin版本为1.9.2,这是一个经过广泛验证且功能强大的编程语言版本。为了与Kotlin兼容,我们选择了Java 1.8作为JRE版本,以确保项目的稳定性和兼容性。

\section{网络请求处理}

在Android开发中,网络请求是一个常见的需求。然而,从Android API 14(即Android 4.0)开始,为了提升用户体验和防止应用卡顿,系统不允许在UI线程执行耗时的网络请求。同时,传统的AsyncTask在Android 12中已被弃用。因此,我们采用了Kotlin语言的协程特性来处理异步网络请求。这种方式不仅简化了代码结构,还提高了应用的响应速度和用户体验。在处理网络请求时,我们将获取的数据转换为Android的Bitmap对象,以便在UI中展示。

\section{接口设计与实现}

该项目的核心功能仍然依赖于OpenCV和NCNN这两个强大的图像处理库。除了直接使用这些库进行图像处理外,我们还通过JNI相关函数将Android平台上的Bitmap对象转换为OpenCV的cv::Mat对象,以便进行进一步的图像处理操作。这样的设计使得我们能够在不改变原有C++代码的基础上,充分利用Android平台的功能和优势。

由于Java语言本身不支持元组输出,这给我们的接口设计带来了一定的挑战。为了解决这个问题,我们将C++中输出的std::tuple类型转换为Java中的Object[]数组类型。这种转换方式虽然增加了一定的复杂性,但确保了Java代码能够正确地处理和使用C++的输出结果。通过精心的接口设计和实现,我们成功地实现了C++与Java之间的无缝对接,使得整个项目能够高效地运行和扩展。

\section{Android与VC环境的差异}

在Android客户端的实现中,一个关键的区别在于模型权重文件的处理方式。不同于传统的VC环境,在Android应用中,模型权重文件是直接被打包进APK文件的。这意味着我们不能像在计算机上那样简单地通过文件路径来访问这些文件。然而,幸运的是,腾讯在开发NCNN时就已经考虑到了移动端的需求,特别是Android操作系统的特性。

NCNN作为一个专门为移动端设计的推理框架,对Android的支持非常出色。具体来说,NCNN提供了"load\_param"和"load\_model"这两个函数,用于在Android平台上加载模型。尽管这两个函数的名称与PC端使用的相同,但在Android NDK环境中,它们的参数有所不同。在Android上,这两个函数接受两个参数:第一个参数是AAssetManager对象的指针,它提供了对APK中assets文件夹的访问;第二个参数则是模型文件的名称。通过这种方式,我们可以轻松地从APK的assets文件夹中加载模型权重文件,实现了在Android平台上的无缝集成。

\section{使用说明}

\begin{enumerate}
	\item 启动程序:点击Luncher上的“上海海事大学验证码识别Demo”,见图\ref{fig:androiddesktop},进入主界面,见图\ref{fig:activitymain}。
	\item 打开图片:“从网络获取验证码”或“本地相册选图”或使用内置图片1和2
	\item 执行OCR:您可以选择使用“CPU识别”或“GPU识别”
\end{enumerate}

\section{运行截图}

APP的桌面启动器图标如图\ref{fig:androiddesktop}所示,而APP的运行界面则如图\ref{fig:activitymain}所展示。关于APP的详细信息,请参考图\ref{fig:appinfo}。值得一提的是,由于我们采用了FP16模型进行优化,使得移动端程序的体积得以显著缩小,仅为117MB。APK文件的体积更是精简至107MB。尽管APK中包含了'x86'、'x86\_64'、'armv7'和'arm64'这四种ABI,导致了体积的相对增大,但我们仍然努力实现了体积的最小化。此外,从Android 5.0版本开始,系统采用了ART虚拟机,这意味着在安装过程中会对程序进行编译优化,因此最终的程序体积达到了117MB。这些优化措施不仅提升了用户体验,还确保了APP的高效运行。

\begin{figure}
	\centering
	\includegraphics[width=0.6\linewidth]{Resources/Picture/Deploy/Android/android_desktop}
	\caption{Android APP 桌面图标}
	\label{fig:androiddesktop}
\end{figure}

\begin{figure}
	\centering
	\includegraphics[width=0.6\linewidth]{Resources/Picture/Deploy/Android/activity_main}
	\caption{Android APP 主界面}
	\label{fig:activitymain}
\end{figure}


\begin{figure}
	\centering
	\includegraphics[width=0.6\linewidth]{Resources/Picture/Deploy/Android/app_info}
	\caption{Android APP 信息}
	\label{fig:appinfo}
\end{figure}



	\chapter{基于Qt C++的跨平台(Windows/macOS/Linux)方案研究}
\label{chapter:11}

\section{介绍}

\section{Windows下进行开发}

\section{适配macOS}

\subsection{开发环境}

\subsection{生成macOS的icns文件}

\section{适配Linux}

\subsection{开发环境}

\begin{enumerate}
	\item Ubuntu 23.10
	\item gcc 12.3.0
	\item Qt 6.6.2
	\item Qt Creator 12.0.2
	\item CLion
	\item CMake
\end{enumerate}

\subsection{界面展示}

	%%%%%%%%%%%%%%%%%%%%%%%%%%%%%%%%%%%%%%%%%%%%%%%%%%%%%%%
	% 总结
	\chapter{总结与展望}
\label{chapter:12}

\section{总结}

\section{展望}

\section{用到的软件}

\subsection{开源软件}

\begin{enumerate}
	\item SHMTU Course LaTex Template(Haomin Kong)\cite{shmtu_course_latex_template}
	\item PyTorch\cite{pytorch}
	\item Digital-SHMTU-Tools\cite{digit_shmtu}
	\item Selenium
	\item Protobuf
	\item Tencent NCNN
	\item OpenVINO
\end{enumerate}

\subsection{商业软件}

\begin{enumerate}
	\item Jetbrains Pycharm
	\item Visual Studio
\end{enumerate}


\end{spacing}

% 参考文献
\newpage
% 下面一行会导入 bib 文件中的所有文献!
% \nocite{*}
\bibliographystyle{abbrv}
\addcontentsline{toc}{chapter}{\heiti 参考文献}
%\addcontentsline{toc}{section}{\heiti 参考文献}
\bibliography{
	References/cv_finalwork.bib,
	References/ref_url.bib
}
\appendix

\end{document}