\newpage
\chapter{概览}

\section{文献选择}

本总结报告主要包含两个研究方向的论文,分别为《视频稳定》和《海平线识别》两个研究方向,其中视频稳定均为在线视频稳定,或者称为实时视频稳定。
一共15篇文献,但是其中额外参考了几个其他的文献,因为原文有所使用或者综述性文献帮助我了解该领域。

\section{实验}

以下论文我进行过实验:

\subsection{海平线识别}

InfML-HDD(第\ref{chapter:1}章),实验详情见\ref{section:1_exam}。

下一步计划请查看\ref{section:next_plan}。

\subsection{视频稳定算法}

\begin{enumerate}
	\item 实时CNN稳定网络StabNet(第\ref{chapter:2stabnet}章),训练显存32GB,推理显存8GB,做不了训练。
	\item 实时Selfie自拍稳定(第\ref{chapter:5selfie}章),推理显存9GB,做不了训练。
	\item DIFRINT光流插帧(第\ref{chapter:3difrint}章),推理显存9GB,做不了训练。
	\item DUT(第\ref{chapter:8dut}章),推理显存10GB,做不了训练。
\end{enumerate}

经过多个实验,我发现我们实验室不具备《视频稳定》研究方向的实验条件。
无论是存储器(没有固态硬盘)还是GPU的VRAM以及GPU浮点计算性能,该课题IO瓶颈极大,多方面远远落后,因此无法继续进行该课题。

\section{为什么在线视频稳定方法比离线方法更具挑战性}

在线视频稳定方法相较于离线方法更具挑战性,主要原因有以下几点:

\begin{enumerate}
	\item 无法获取未来帧:在线方法在优化过程中仅能访问过去和当前的帧,而离线方法可以在整个视频捕捉后进行全局路径优化,这使得在线方法面临更大的挑战。
	\item 实时性要求:在线方法需要在实时捕捉过程中稳定视频,同时保持高效的计算性能,而离线方法可以在捕捉后进行优化,不受实时性限制。
	\item 复杂运动处理:在线方法需要应对快速旋转、变焦等具有挑战性的相机运动,这些运动在全局路径优化中可以得到满意的解决方案,但在线方法需要通过深度在线相机路径优化框架解决这些问题。
\end{enumerate}

综上所述,在线视频稳定方法需要在实时性和计算效率方面进行权衡,并应对各种具有挑战性的相机运动,这使得它们比离线方法更具挑战性。