\chapter{背景及意义}
\label{chapter:1}

\section{背景}

2023年秋季,我踏入了上海海事大学的校园,感受着这所学术殿堂的独特氛围。在日常的校园生活中,微信小程序“上海海事大学一卡通”成为了我管理个人账务的得力助手。这款小程序界面简洁明了,功能齐全,尤其是最下方的三个选项卡中的“记录”选项,它详尽地展示了我的每一笔消费记录,如图\ref{fig:wechatbill}所示。这样的设计不仅方便我随时掌握自己的消费情况,还激发了我进一步探索记账领域的兴趣。

然而,在尝试通过网页平台进行更深入的账务管理时,我遇到了一些技术上的挑战。由于平台采用了统一认证接口进行用户身份验证,我的Selenium自动化程序虽然能够自动填充登录信息,但却无法自动处理验证码识别这一环节。这意味着每次运行爬虫程序时,浏览器窗口都会以非headless模式(即用户界面可见)打开,并等待我手动输入验证码。这样的操作流程不仅繁琐,而且无法实现全自动化,给我的账务管理带来了一定的不便。

为了解决这个问题,我开始探索各种可能的解决方案。我研究了验证码识别的相关技术,尝试了多种图像处理和机器学习的方法,以期能够开发出一种能够自动识别验证码的程序。虽然这个过程充满了挑战,但我相信通过不断的学习和实践,我一定能够克服这些困难,实现账务管理的全自动化。同时,我也期待着在未来的学习和工作中,能够将这些技术和经验应用到更广泛的领域,为校园生活和社会发展贡献自己的力量。

\begin{figure}
	\centering
	\includegraphics[width=0.6\linewidth]{Resources/Picture/wechat_bill}
	\caption{微信查询校园卡消费账单}
	\label{fig:wechatbill}
\end{figure}

2024年1月,学期末的考试如期而至,我也在紧张的备考后迎来了一个小小的间歇。应导师的深情“建议”,我决定留校继续学习一个月,以巩固所学知识并进一步深化研究。那段时间,我每天都沉浸在实验室的研究中,常常忙碌到深夜才踏着月色回到宿舍。

有一天,我结束实验回到宿舍时已经夜深人静,时钟的指针早已越过了12点。疲惫不堪的我打算洗个热水澡放松一下身心,然而却发现洗澡水的温度并不高,甚至可以说是有些凉意。在那样一个气温零下的寒夜,这无疑是一种令人沮丧的体验。

我忍不住向师兄抱怨起这件事情,他听后微笑着拿出手机,打开了“海大后勤”微信公众号。他告诉我,这个公众号不仅提供了校园内的各种生活服务信息,还能实时查看每栋宿舍楼当前的洗澡水温度。我好奇地凑过去看,只见屏幕上显示着一个清晰的图表,详细标注了各个宿舍楼的洗澡水温度数据,如图\ref{fig:wechathotwater}所示。

师兄还告诉我,如果我想获取更详细的数据或者进行一些个性化的设置,可以通过抓取相关的URL接口来实现。这立刻引起了我的兴趣,作为一名技术爱好者,我决定尝试一下。我拿出自己的小米手机,利用Root权限开始了抓包操作。经过一番努力,我终于成功抓到了所需的URL接口。

然而,就在我准备进一步探索这个接口的功能时,却发现它同样需要进行统一认证才能访问。这意味着我需要先通过学校的身份验证系统才能获得访问权限。虽然有些失望,但我并没有放弃。我相信通过不断的学习和实践,我一定能够找到解决这个问题的方法,让洗澡水温度的查询变得更加便捷和智能化。同时,我也期待着在未来的学习和生活中,能够将这些技术和经验应用到更广泛的领域,为校园生活和社会发展贡献自己的力量。

\begin{figure}
	\centering
	\includegraphics[width=0.6\linewidth]{Resources/Picture/wechat_hot_water}
	\caption{微信查询洗澡热水水温}
	\label{fig:wechathotwater}
\end{figure}

而且,值得一提的是,小米手机从MIUI 12版本开始,在负一屏为用户提供了各种便捷的卡片式信息展示。这些卡片不仅美观实用,而且能够实时更新各种生活信息,如天气、日程、快递等。我深受启发,认为将来完全可以结合这个桌面小组件的功能,进一步方便我们获取洗澡水的温度信息。想象一下,只需轻轻滑动到负一屏,就能一目了然地看到当前宿舍楼的洗澡水温度,这将极大地提升我们的校园生活体验。

说到校园生活,其实我的学术生涯也与之紧密相连。我们课题组的研究方向主要是利用深度学习的方法进行图像的目标检测与增强处理。为了更深入地掌握这一领域的知识,我特意选修了王建华老师的《机器视觉》课程,以及付广华老师的《Python程序设计》和《人工智能》课程。在科研的过程中,我不断地将所学知识应用到实际项目中,这不仅让我对理论知识有了更深刻的理解,也让我在实践中锻炼了自己的动手能力。

现在,我准备利用我所学的知识来解决统一认证接口的最后一道难题——验证码识别。我深知这是一个充满挑战的任务,因为验证码的设计初衷就是为了防止自动化程序的攻击。然而,我也坚信通过不断的学习和实践,我一定能够找到一种有效的方法来突破这个限制。也许我可以尝试利用图像处理和机器学习的技术来自动识别验证码;也许我可以借鉴其他领域的经验和方法来寻找新的突破口。无论如何,我都将全力以赴,为实现全自动化的账务管理贡献自己的力量。

\section{意义}

这个项目不仅是我对王建华老师的《机器视觉》课程和付广华老师的《Python程序设计》和《人工智能》课程知识的综合运用,更代表着我个人在深度学习领域的学术探索与实践,同时也是我多年计算机程序设计经验的一次全面复习。在这个过程中,我重新回顾了多个曾经熟悉的项目,这些项目涉及不同的领域和技术,让我有机会再次巩固和加深了对计算机程序设计的理解。

此外,我还运用了自己制作的LaTeX模板来整理和呈现项目成果,这不仅提高了我的工作效率,也让我更加熟悉和掌握了LaTeX这一强大的排版工具。通过这个项目,我不仅复习了过往的知识和经验,更重要的是,我解决了自己在实际应用中遇到的一些痛点和难题。这些问题的解决,不仅提升了我的技术水平,也让我更加自信和坚定地走向未来的学习和工作。

同时,我深知学术界的理论研究和工业界的实际应用之间存在着巨大的鸿沟。尽管我在理论层面已经有了一定的积累,但要将这些知识应用到实际工作中,还需要掌握更多实践技能。特别是在模型的量化和部署方面,这是我之前未曾涉足的领域,但却是工业界非常关注的重要环节。因此,我希望通过这个项目,能够将我在课堂、科研以及多年程序设计中积累的知识真正应用到实践中,特别是在模型的量化和部署方面取得突破。

模型的量化与部署是将深度学习技术从理论推向实际应用的关键环节。通过这次实践,我将学习并掌握相关的技术和工具,了解工业界的实际需求和应用场景。这将使我更全面地掌握深度学习的全流程,提高我的实践能力和综合素质,为我未来的职业发展打下坚实的基础。

总的来说,这个项目不仅是我对过往学习成果的一次总结,更是我对未来职业发展的一次有力探索。它让我有机会复习和巩固了多年的计算机程序设计经验,解决了我在实际应用中遇到的痛点,同时也让我更加熟悉和掌握了深度学习领域的先进技术和实践方法。我相信,在未来的学习和工作中,我将以更加扎实的知识储备和更加自信的态度,迎接更多的挑战和机遇。

\section{预备知识}

\begin{enumerate}
	\item OpenCV机器视觉图像处理
	\item 深度学习
	\item KMeans聚类算法
	\item Python语言(Python 3.8+)
	\item C++语言(C++20标准)
	\item Microsoft Visual C++(VC)语言
	\item Windows API
	\item C\# 8.0语言
	\item Java SE(1.8)
	\item Kotlin语言
	\item Android
	\item Gradle构建脚本
	\item PyTorch框架
	\item HTML基础
	\item Selenium自动化测试框架
	\item ONNX推理框架
	\item NCNN推理框架
\end{enumerate}

\section{环境要求}

Windows平台以及Android平台都有方案兼容10年前以前的操作系统,因此无需担心兼容性。

\subsection{MSVC版本}

经过我的测试Windows 7(发布于2019年)可以完美运行。
此外VC运行库官网标注支持Vista操作系统,因此XP以及以上版本操作系统理论上能够正常运行。

\subsection{Qt版本}

Qt支持Windows 7以上版本,但是并不支持Windows 7,而且可能需要安装Direct X图形库。

\subsection{.Net版本}

可能是需要至少Windows 10操作系统。

\subsection{Android平台}

最低运行操作系统版本为Android API 21即Android 5.0(Lollipop)。
API 21发布于2014年,恰好为10年前,这个版本使用ART虚拟机代替了Dalvik虚拟机,因此我选择这个版本作为最低版本。

\section{开发(参考)环境}

下面给出我使用的开发环境供参考。

\subsection{Windows}

硬件:
\begin{enumerate}
	\item Intel Core i5-11400
	\item AMD Radeon RX 580 2048SP
	\item NVIDIA Tesla P40
\end{enumerate}

软件:
\begin{enumerate}
	\item Windows 11 23H2
	\item Visual Studio 2022
	\item cl.exe
	\item CMake
	\item TeXLive 2023
	\item Jetbrains PyCharm
\end{enumerate}

\subsection{Linux}

硬件:
\begin{enumerate}
	\item Intel Core i5-11400
	\item AMD Radeon RX 580 2048SP
	\item NVIDIA Tesla P40
\end{enumerate}

软件:
\begin{enumerate}
	\item Ubuntu 23.10
	\item Linux Kernel 6.5.0-17-generic
	\item KDE Plasma 5.27.8
	\item 图形平台:X.Org(X11)
	\item gcc-12
	\item g++-12
	\item Android Studio Hedgehog 2023.1.1 Patch 2
	\item CMake 3.28.1
	\item Android NDK
	\item Jetbrains PyCharm 2023.3.3
	\item Qt 6.6.2
	\item Jetbrains CLion 2024.1 EAP (Nova)
\end{enumerate}

\subsection{macOS}

硬件:
\begin{enumerate}
	\item Intel Core i7-8850H
	\item AMD Radeon Pro RX 560X
\end{enumerate}

软件:
\begin{enumerate}
	\item macOS 14.2.1
	\item Apple LLVM Clang
	\item CMake 3.28.1
	\item Jetbrains PyCharm 2023.3.3
	\item Qt 6.6.2
	\item Jetbrains CLion 2024.1 EAP (Nova)
\end{enumerate}
