\chapter{Python 推理}
\label{chapter:7}

在Python中进行深度学习模型的推理是一个重要的环节,它涉及到将训练好的模型应用于实际数据以获取预测结果。本章节将详细介绍使用不同工具和框架进行模型推理的方法。

\section{PyTorch直接推理}

PyTorch作为一个流行的深度学习框架,提供了方便的工具来进行模型的推理。通过调用model.eval()方法,我们可以将模型设置为评估模式,此时模型的参数将不再更新,适用于进行推理测试。在评估模式下,我们可以利用训练好的模型对新的数据进行预测,并获取相应的输出结果。

具体代码请参考\url{Train/SHMTU_CAS_OCR_RESNET/src/classify/predict/predict_file.py}。

\section{ONNX Runtime}

ONNX(Open Neural Network Exchange)是一种开放的深度学习模型表示格式,得到了众多深度学习框架的支持。通过将模型导出为ONNX格式,我们可以利用ONNX Runtime在不同平台和设备上进行模型的推理。ONNX Runtime是一个高效的推理引擎,支持多种硬件加速,包括CPU、GPU和专用加速器等。通过使用ONNX Runtime,我们可以方便地在不同环境中部署和推理模型。

具体的ONNX Runtime推理代码可以参考位于\url{Train/SHMTU_CAS_OCR_RESNET/src/classify/digit/infer_onnx.py}路径下的代码文件。

\section{Intel OpenVINO}

考虑到大部分PC和服务器使用的是Intel处理器,而且Intel还提供了神经计算棒等外接设备,可以方便地扩展嵌入式设备如树莓派的深度网络推理能力,我们选择使用Intel OpenVINO工具套件进行推理。OpenVINO是Intel推出的一个优化推理性能的开源工具,可以充分利用Intel硬件的优势,提供高效的推理性能。

在市场上,二手Intel神经计算棒的价格相对较低,这使得开发者可以轻松地获取到这些设备进行开发和测试。图\ref{fig:intelstick}展示了Intel神经计算棒的外观。

\begin{figure}
	\centering
	\includegraphics[width=0.5\linewidth]{Resources/Picture/intel_stick}
	\caption{Intel神经计算棒}
	\label{fig:intelstick}
\end{figure}

具体使用Intel OpenVINO进行推理的代码可以参考\url{Train/SHMTU_CAS_OCR_RESNET/src/classify/digit/openvino_test.py}路径下的代码文件。通过这些代码,我们可以了解如何加载模型、设置输入数据以及执行推理等关键步骤。

我们选择Python进行原型设计,但是OpenVINO也提供了C++ API,在将来我们也可以很方便地将程序改写为C++以提升运行效率。
