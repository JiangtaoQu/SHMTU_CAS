\chapter{问题分析}
\label{chapter:2}

\section{验证码爬虫}

上海海事大学统一认证平台,界面见图\ref{fig:caswebpage}。该平台的接口URL为\url{https://cas.shmtu.edu.cn/cas/login},然而其验证码的URL为\url{https://cas.shmtu.edu.cn/cas/captcha},每次访问这个网址都可以获取到一个400x140x3的彩色PNG图片,因此我们只需要使用Python的Requests库进行爬虫即可,间隔3秒下载1次,保存为当前时间命名的文件中。

\begin{figure}
	\centering
	\includegraphics[width=0.9\linewidth]{Resources/Picture/cas_webpage}
	\caption{上海海事大学统一认证平台}
	\label{fig:caswebpage}
\end{figure}

\section{验证码图片分析}

验证码图片示例见图\ref{fig:test120240102160004server}和图\ref{fig:test220240102160811server}。

验证码图片分为4部分:

\begin{enumerate}
	\item 数字1
	\item 运算符号
	\item 数字2
	\item 等于号
\end{enumerate}

其中,第1部分与第3部分的数字均能保证为阿拉伯数字,然而第2部分和第4部分均有可能为汉字或符号,而且验证码的每一部分经常连接在一起,因此不方便进行整体OCR。

\begin{figure}
	\centering
	\includegraphics[width=0.7\linewidth]{Resources/Picture/test1_20240102160004_server}
	\caption{验证码图片-示例1}
	\label{fig:test120240102160004server}
\end{figure}

\begin{figure}
	\centering
	\includegraphics[width=0.7\linewidth]{Resources/Picture/test2_20240102160811_server}
	\caption{验证码图片-示例2}
	\label{fig:test220240102160811server}
\end{figure}

\section{解决方案}

\subsection{Paddle OCR}

Paddle OCR是一个基于百度Paddle Paddle深度学习框架的开源OCR工具库,旨在提供高效、易用且灵活的OCR解决方案。它包含了多种功能和模型,可以用于文字检测、文字识别、文本方向检测、多语言OCR等任务。

项目地址:\url{https://github.com/PaddlePaddle/PaddleOCR}

以下是Paddle OCR的主要特点和功能:

\begin{enumerate}
	\item 丰富的模型支持:Paddle OCR提供了多个预训练的模型,涵盖了文字检测、文本识别等多个任务。这些模型具有不同的轻量级和高精度的特点,可以根据具体需求进行选择。
	\item 多任务支持:Paddle OCR支持多种OCR任务,包括文本检测、文本识别、文本方向检测等。用户可以根据需求选择不同的任务模型,并将它们组合在一起构建完整的OCR系统。
	\item 易用性:Paddle OCR提供了简单易用的Python接口,用户可以通过几行代码就能实现文字检测和识别的功能。同时,它还提供了丰富的文档和示例代码,帮助用户快速上手。
	\item 高性能:Paddle OCR基于Paddle Paddle深度学习框架,利用其高性能的计算能力和优化的算法,能够在各种硬件平台上实现高效的文字检测和识别。
	\item 灵活性:Paddle OCR提供了灵活的模型配置和训练接口,用户可以根据自己的需求进行模型的定制和优化。同时,它还支持多种数据格式和输入方式,适应不同场景和需求。
\end{enumerate}

总的来说,Paddle OCR是一个功能丰富、易用高效的OCR工具库,适用于各种文字检测和识别任务,可以帮助用户快速构建和部署OCR系统。

我尝试使用Paddle OCR进行识别,然而识别结果总是不尽如人意,因为他每一部分连接在一起,因此很容易将多部分认为是一个字符,而且很容易将减号识别为中文数字“一”,因此无法使用Paddle OCR进行处理。

\subsection{分片段识别}

为了更精确地识别数学表达式中的各个部分,我首先进行了大规模的图片爬取,累计获取了5万张相关的图像数据。接着,我对这些图片执行了以下操作:

\begin{enumerate}
	\item 分类处理:我根据图片中“等于”和“=”这两个符号的出现情况,对图像进行了分类。这样做的目的是为了区分包含等号和包含“等于”二字,为后续的处理提供便利。
	\item 等号去除:对于包含等号的图像,我进一步根据比例关系去除了等号部分,只保留了纯计算式的区域。这一步骤的目的是为了提取出不含等号的数学表达式,便于后续的识别工作。
	\item 尺寸调整与分割:考虑到不同数学表达式在图像中可能呈现出不同的尺寸,我采取了按比例分割的方法。通过对图像中数学表达式的尺寸进行分析,我将其分割成适合识别的片段。
\end{enumerate}

通过上述步骤,我成功地实现了对计算式的有效分割。
实验结果表明,这种分段识别的方法能够显著提高识别的准确率。因此,我决定将分割后的四个部分分别进行独立的识别处理,以进一步提升整体识别效果。
