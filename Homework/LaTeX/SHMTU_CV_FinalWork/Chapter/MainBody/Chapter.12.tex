\chapter{分析与展望}
\label{chapter:12}

\section{内存占用分析}

为了对比这几种GUI框架的内存占用,我统计了每种GUI的每种情况下的内存占用。
MSVC版本的内存占用情况见表\ref{tab:memory_msvc},.Net WPF版本的内存占用情况见表\ref{tab:memory_dotnet},Qt Windows版本的内存占用情况见表\ref{tab:memory_qt_windows}。

\begin{table}[!ht]
	\centering
	\caption{MSVC内存占用}
	\begin{tabular}{cc}
		\hline
		MSVC & 内存占用(MB) \\ \hline
		程序运行 & 22.3 \\
		CPU加载模型 & 716.3 \\
		GPU加载模型 & 82.5 \\
		GPU加载模型的峰值 & 334.5 \\ \hline
	\end{tabular}
	\label{tab:memory_msvc}
\end{table}

\begin{table}[!ht]
	\centering
	\caption{.Net WPF内存占用}
	\begin{tabular}{cc}
		\hline
		.Net WPF & 内存占用(MB) \\ \hline
		程序运行 & 74.5 \\
		CPU加载模型 & 778.6 \\
		GPU加载模型 & 151.5 \\
		GPU加载模型的峰值 & 415.3 \\ \hline
	\end{tabular}
	\label{tab:memory_dotnet}
\end{table}

\begin{table}[!ht]
	\centering
	\caption{Qt Windows内存占用}
	\begin{tabular}{cc}
		\hline
		Qt & 内存占用(MB) \\ \hline
		程序运行 & 26.1 \\
		CPU加载模型 & 722.2 \\
		GPU加载模型 & 89.2 \\
		GPU加载模型的峰值 & 574.3 \\ \hline
	\end{tabular}
	\label{tab:memory_qt_windows}
\end{table}

根据这三个表格,我们可以分析这些GUI框架在MSVC、.Net WPF和Qt Windows版本下的内存占用情况。

\subsection{MSVC版本}

\begin{itemize}
	\item 程序运行:内存占用为22.3MB,这是相对较低的。
	\item CPU加载模型:内存占用增加到716.3MB,这可能是因为CPU密集型任务需要更多的内存来处理数据。
	\item GPU加载模型:内存占用为82.5MB,比程序运行时的内存占用略高,但远低于CPU加载模型。这可能是因为GPU密集型任务更多地依赖于图形处理,而不是内存。
	\item GPU加载模型的峰值:内存占用达到334.5MB,这可能是因为GPU在处理任务时突然需要更多的内存。
\end{itemize}

\subsection{.Net WPF版本}

\begin{itemize}
	\item 程序运行:内存占用为74.5MB,比MSVC版本高。
	\item CPU加载模型:内存占用高达778.6MB,比MSVC版本更高。这可能是因为.Net WPF在处理CPU密集型任务时比MSVC需要更多的资源。
	\item GPU加载模型:内存占用为151.5MB,比MSVC版本高,但低于CPU加载模型。
	\item GPU加载模型的峰值:内存占用达到415.3MB,这是所有框架中GPU加载模型峰值最高的。
\end{itemize}

\subsection{Qt Windows版本}

\begin{itemize}
	\item 程序运行:内存占用为26.1MB,略高于MSVC版本。
	\item CPU加载模型:内存占用为722.2MB,与.Net WPF版本相近,但略高。
	\item GPU加载模型:内存占用为89.2MB,比MSVC版本高,但低于.Net WPF版本。
	\item GPU加载模型的峰值:内存占用达到574.3MB,这是所有框架中GPU加载模型峰值第二高的。
\end{itemize}

\subsection{内存占用总结}

\begin{itemize}
	\item 在程序运行时,MSVC版本的内存占用最低,而.Net WPF版本的内存占用最高。
	\item 在CPU加载模型下,.Net WPF和Qt Windows版本的内存占用相近,且都高于MSVC版本。
	\item 在GPU加载模型下,MSVC版本的内存占用最低,而.Net WPF版本的内存占用最高。
	\item 在GPU加载模型的峰值下,.Net WPF版本的内存占用最高,Qt Windows版本次之,MSVC版本最低。
\end{itemize}

这些结果表明,对于CPU密集型任务,MSVC版本可能具有更好的内存效率,而对于GPU密集型任务,MSVC版本在内存使用上也表现良好。另一方面,.Net WPF在处理CPU和GPU密集型任务时可能需要更多的内存资源。Qt Windows版本在内存使用上介于MSVC和.Net WPF之间。

Windows平台下如果追求极值的高效,建议选择MSVC作为GUI主框架,如果为了追求较高的开发效率建议使用.Net技术栈,如果为了追求跨平台建议使用Qt。

\section{展望}

在本文所探讨的模型中,我们仅采用了ResNet这一经典的深度网络架构。尽管ResNet已在许多视觉任务中展现了其强大的性能,但随着深度学习技术的不断进步,我们有理由相信,引入更先进、更高效的网络架构将进一步提升模型的预测精度。例如,Apple提出的FastViT(Fast Vision Transformer)\cite{Vasu_Gabriel_Zhu_Tuzel_Ranjan_2023}便是一种值得尝试的先进方法。FastViT将Transformer结构应用于视觉任务,通过自注意力机制对图像进行全局和局部的特征提取,从而有可能在复杂场景下提供更精确的特征表示。

此外,为了提高模型预测的准确度,我们还可以考虑引入更多的分割方法。分割方法提供更精细的图像分析,帮助模型更好地理解和识别图像中的细节信息。

因此,未来的研究方向可以包括探索并整合更多先进的网络架构,如FastViT等Vision Transformer方法,以及结合多种分割方法来提升模型在视觉任务中的预测性能。这将有助于推动深度学习在图像处理领域的进一步发展,实现更准确的视觉感知和智能分析。

\section{工作量}

我大概使用一万四千行代码完成了本项目,耗时一周多,Github统计的代码占比见图\ref{fig:statgithub},Tokei统计的每门语言的代码行数见图\ref{fig:stattokei}。

\begin{figure}
	\centering
	\includegraphics[width=0.9\linewidth]{Resources/Picture/stat_github}
	\caption{Github统计的代码占比}
	\label{fig:statgithub}
\end{figure}

\begin{figure}
	\centering
	\includegraphics[width=0.9\linewidth]{Resources/Picture/stat_tokei}
	\caption{Tokei统计的代码行数(单位:行)}
	\label{fig:stattokei}
\end{figure}

\section{用到的软件}

\subsection{开源软件(仓库)}

\begin{enumerate}
	\item SHMTU Course LaTex Template(Haomin Kong Version)\cite{shmtu_course_latex_template}
	\item Python\cite{python}
	\item PyTorch\cite{pytorch}
	\item Digital-SHMTU-Tools\cite{digit_shmtu}
	\item Selenium\cite{selenium}
	\item Tencent NCNN\cite{ncnn}
	\item OpenVINO\cite{openvino}
	\item TeXLive\cite{texlive}
	\item TeXStudio\cite{texstudio}
	\item OpenCV\cite{opencv}
	\item Vulkan SDK\cite{vulkan-sdk}
	\item ONNX\cite{onnx}
\end{enumerate}

\subsection{商业软件}

\begin{enumerate}
	\item Jetbrains Pycharm\cite{jetbrains}
	\item Jetbrains CLion\cite{jetbrains}
	\item Visual Studio\cite{vs}
\end{enumerate}
