\chapter{Microsoft Visual C++的部署}
\label{chapter:8}

\section{NCNN API}

\subsection{工作流程}

ncnn是一个为移动端设计的高效神经网络前向计算框架。以下是使用ncnn的C++ API进行神经网络推断的基本步骤:

\begin{enumerate}

	\item \textbf{包含必要的头文件}

	在使用ncnn之前,需要包含必要的头文件。这通常包括ncnn的主要命名空间和一些辅助工具。

	\begin{lstlisting}[caption={导入OpenCV与NCNN},language=C++]
		#include <net.h>
		#include <opencv2/opencv.hpp>
	\end{lstlisting}

	\item \textbf{加载预训练模型和参数}

	ncnn可以从硬盘加载已经转换好的预训练模型,这些模型通常由其他框架(如Caffe、TensorFlow等)转换而来。

	\begin{lstlisting}[caption={加载权重文件},language=C++]
		ncnn::Net net;
		net.load_param("model.param");
		net.load_model("model.bin");
	\end{lstlisting}

	\item \textbf{预处理输入数据}

	根据模型的输入要求,对输入数据进行预处理。这可能包括缩放、裁剪、归一化等操作。

	\begin{lstlisting}[caption={读取图片并处理数据},language=C++]
		cv::Mat img = cv::imread("input.jpg");
		ncnn::Mat in = ncnn::Mat::from_pixels_resize(img.data, ncnn::Mat::PIXEL_BGR, img.cols, img.rows, model_input_width, model_input_height);
		in.substract_mean_normalize(mean_vals, norm_vals);
	\end{lstlisting}

	请注意,`mean\_vals` 和 `norm\_vals` 需要根据你的模型设置正确的均值和标准差。

	\item \textbf{创建提取器并设置输入}

	创建一个提取器对象,并将预处理后的数据设置为输入。

	\begin{lstlisting}[caption={设置输入输出},language=C++]
		ncnn::Extractor ex = net.create_extractor();
		ex.input("input_blob", in);
	\end{lstlisting}

	这里的`"input\_blob"`是模型中定义的输入层的名称,需要根据实际情况进行替换。

	\item \textbf{执行推断}

	调用提取器的`extract`方法来执行推断。

	\begin{lstlisting}[caption={推理},language=C++]
		ncnn::Mat out;
		ex.extract("output_blob", out);
	\end{lstlisting}

	这里的`"output\_blob"`是模型中定义的输出层的名称,需要根据实际情况进行替换。

	\item \textbf{后处理输出数据}

	根据需要对输出数据进行后处理,例如将结果解码为可读的格式或应用于其他任务。

\end{enumerate}

这些步骤提供了使用ncnn进行神经网络推断的基本框架。具体细节可能因模型和数据而异,需要根据具体情况进行调整。

\subsection{NCNN调用Vulkan进行GPU加速}

ncnn是一个高效的神经网络前向计算框架,支持使用Vulkan进行加速。Vulkan是一个跨平台的计算机图形和计算API,通常用于游戏和图形应用程序,但也可以用于加速神经网络的推理。

使用Vulkan加速ncnn的一般步骤如下:

\begin{enumerate}
	\item 安装Vulkan:首先,你需要在你的系统上安装Vulkan。这通常涉及到下载和安装Vulkan SDK,并配置相关的环境变量。具体的安装步骤可能因操作系统和硬件的不同而有所不同。
	\item 编译ncnn with Vulkan:在安装了Vulkan之后,你需要使用Vulkan来编译ncnn。这通常涉及到在编译ncnn时启用Vulkan支持。具体的编译步骤可能因ncnn的版本和你的开发环境而有所不同。
	\item 使用编译好的ncnn进行推理:一旦你使用Vulkan编译了ncnn,你就可以使用编译好的ncnn来进行神经网络的推理了。在推理时,ncnn会自动使用Vulkan进行加速。
\end{enumerate}

需要注意的是,使用Vulkan加速ncnn并不总是能提供显著的性能提升。这取决于多种因素,包括你的硬件、操作系统、ncnn的版本以及你正在运行的神经网络模型。在某些情况下,使用Vulkan可能会降低性能,因此在使用之前最好进行充分的测试和性能分析。

另外,ncnn还支持其他加速方式,如使用ARM NEON指令集进行加速等。你可以根据你的硬件和需求选择最适合的加速方式。

\begin{lstlisting}[caption={NCNN with Vulkan Initialization}, language=C++]
	#include "net.h"
	// ... 其他必要的NCNN和Vulkan头文件 ...

	int main() {
		// 初始化NCNN网络,并启用Vulkan支持
		ncnn::Net net;

		// 启用Vulkan计算
		net.opt.use_vulkan_compute = true;

		// 加载模型
		net.load_param("model.param");
		net.load_model("model.bin");

		// 准备输入数据(伪代码)
		// ncnn::Mat input = prepareInputData(...);

		// 运行网络进行推理
		// ncnn::Extractor extractor = net.create_extractor();
		// extractor.input("input_blob", input);
		// extractor.extract("output_blob", output);

		// 处理输出数据(伪代码)
		// processOutputData(output);

		return 0;
	}
\end{lstlisting}

\section{Visual Studio 解决方案}

\begin{figure}
	\centering
	\includegraphics[width=0.7\linewidth]{Resources/Picture/Deploy/Windows/vs}
	\caption{Visual Studio解决方案Project}
	\label{fig:vs}
\end{figure}

\begin{enumerate}
	\item SHMTU\_CAS\_OCR\_Demo\_VC
	\item NCNN\_CLI
	\item SHMTU\_CAS\_CLR\_ClassLibrary
	\item SHMTU\_CAS\_OCR\_CLR\_Connector\_ClassLibrary
	\item SHMTU\_CAS\_OCR\_Demo\_WPF
\end{enumerate}

\subsection{SHMTU\_CAS\_OCR\_Demo\_VC}

\subsection{NCNN\_CLI}

\subsection{SHMTU\_CAS\_CLR\_ClassLibrary}

\subsection{SHMTU\_CAS\_OCR\_CLR\_Connector\_ClassLibrary}

\subsection{SHMTU\_CAS\_OCR\_Demo\_WPF}

\section{第三方库}

\subsection{NCNN20240102}

Github上提供的预先构建的NCNN Release版本采用了Release构建类型,这种类型主要优化的是程序的运行速度,同时会去除调试信息以减小文件大小。然而,这样的构建版本并不适合在Visual Studio(VC)中进行调试,因为它不包含必要的调试符号和信息。为了能够在VC中顺利进行NCNN的调试工作,我决定直接从Github仓库中克隆原始的NCNN代码,并在本地使用适当的配置重新编译它,以确保生成的库文件包含完整的调试信息。

\subsection{Protobuf 3.11.2}

ONNX(Open Neural Network Exchange)是一种用于表示深度学习模型的开放标准,它使用Google的Protobuf(Protocol Buffers)来进行模型数据的序列化和反序列化。因此,要在使用NCNN处理ONNX模型时,Protobuf是一个必不可少的依赖项。根据NCNN的官方文档推荐,Protobuf 3.11.2版本与NCNN的兼容性较好。基于这一信息,我选择了这个版本进行编译,以确保后续工作的顺利进行。

\subsection{OpenCV 5.0.0}

在已经决定自己编译Protobuf和NCNN的情况下,我考虑到OpenCV(Open Source Computer Vision Library)也是一个经常与NCNN一起使用的库,而且它的最新版本可能会带来一些新的功能和性能提升。虽然当前的稳定版本是4.9.0,但OpenCV的5.x版本已经在开发过程中,并且其源代码已经在Github上公开。因此,我决定直接从Github上克隆OpenCV的5.x开发分支,并在本地进行编译,以生成最新的5.0.0版本。这样一来,我就可以在我的项目中使用到最新版本的OpenCV,同时确保它与我自己编译的NCNN和Protobuf库之间的兼容性。

\section{编译步骤}

\subsection{安装Visual Studio 2022}

\begin{enumerate}
	\item 前往Visual Studio官网(\url{https://visualstudio.microsoft.com/})下载Visual Studio 2022安装程序。
	\item 运行安装程序,并根据提示选择需要的组件,例如Visual C++等。
	\item 等待安装完成,确保安装过程中没有错误。
\end{enumerate}

\subsection{安装CMake}

\begin{enumerate}
	\item 前往CMake官网(\url{https://cmake.org/})下载适用于Windows的CMake安装程序。
	\item 运行安装程序,并按照默认设置进行安装。
	\item 将CMake添加到系统路径中,以便在命令行中使用。
\end{enumerate}

\subsection{安装Vulkan SDK}

本程序使用了Vulkan进行GPU加速,因此必须安装Vulkan SDK。

\begin{enumerate}
	\item 前往Vulkan SDK的下载页面(\url{https://vulkan.lunarg.com/sdk/home})下载适用于Windows的最新版Vulkan SDK安装程序。
	\item 运行安装程序,并按照提示进行安装。通常情况下,安装程序会自动配置环境变量和路径。
	\item 验证安装是否成功,可以通过检查Vulkan SDK的安装目录和相关的环境变量来确认。
\end{enumerate}

\subsection{编译NCNN}

\begin{enumerate}
	\item 前往NCNN的GitHub仓库(\url{https://github.com/Tencent/ncnn})克隆源代码到本地计算机上。
	\item 打开命令行工具,进入NCNN源代码目录。
	\item 使用CMake生成构建文件,指定需要的编译器和其他配置选项。例如,运行命令:\texttt{cmake -DCMAKE\_BUILD\_TYPE=Release ..}。
	\item 使用Visual Studio或MSBuild编译生成的解决方案或项目文件。确保选择了正确的配置(例如Release)和平台(例如x64)。
	\item 编译完成后,将生成的库和头文件放置到适当的位置,以便在项目中使用。
\end{enumerate}

\subsection{编译OpenCV}

我手动编译5.x分支并不只是为了尝鲜5.0.0版本,还一个用途就是编译静态库版本可执行程序文件。

编译步骤如下:

\begin{enumerate}
	\item 前往OpenCV的GitHub仓库(\url{https://github.com/opencv/opencv})并克隆"5.x"分支到本地
	\item 创建一个构建目录,并在命令行工具中进入该目录。
	\item 使用CMake GUI配置OpenCV的构建。
	\item 取消DNN、Java、Python、Protobuf相关用不到的功能选项以加快编译速度。
	\item 根据需要配置"BUILD\_SHARED\_LIBS"用于配置是否构建静态库版本。
	\item 使用Visual Studio或MSBuild编译生成的解决方案或项目文件。这可能需要一些时间,具体取决于您的计算机性能和配置选项。
	\item 编译完成后,您可以选择安装OpenCV库和头文件到系统目录,或者将它们复制到您的项目目录中以便使用。确保在您的项目中正确配置库路径和头文件路径。
\end{enumerate}

\section{CAS\_OCR.cpp}

\subsection{介绍}

"CAS\_OCR.cpp"文件为本项目的核心代码,无论是MSVC版本还是.Net版本还是Qt这种计算机上的版本甚至Android版本都使用到了这个程序,因此我在这个源代码中实现了许多的宏定义用于适配不同的编译器环境以及操作系统。

具体代码请参考\url{Deploy/SHMTU_CAS_OCR_Demo_Windows/VC/NCNN_CLI/CAS_OCR.cpp}。

\subsection{每个函数的功能}

下面为每个函数的功能介绍:

\begin{itemize}
	\item \texttt{split\_img\_by\_ratio}: 根据给定的比例裁剪图像。
	\item \texttt{path\_check\_windows\_style}: 检查路径是否符合 Windows 风格。
	\item \texttt{path\_ensure\_slash}: 确保路径以斜杠结尾。
	\item \texttt{init\_model\_for\_net}: 初始化指定模型的网络。
	\item \texttt{init\_model}: 初始化模型。
	\item \texttt{predict\_by\_model}: 使用模型进行预测。
	\item \texttt{get\_operator\_type\_by\_int}: 根据整数值获取运算符类型(int转换为枚举类)。
	\item \texttt{get\_operator\_str\_by\_int}: 根据整数值获取运算符字符串表示。
	\item \texttt{calculate\_operator}: 执行运算操作。
	\item \texttt{predict\_validate\_code}: 预测并验证码。
	\item \texttt{release\_model}: 释放模型资源。
	\item \texttt{is\_model\_init}: 检查模型是否已初始化。
	\item \texttt{set\_model\_opt\_gpu}: 设置模型在 GPU 上的选项。
	\item \texttt{set\_model\_gpu\_support}: 设置模型是否支持 GPU。
	\item \texttt{NCNN\_Device\_Info}: 表示 NCNN 设备信息的类。
	\item \texttt{print\_device\_info}: 打印设备信息。
	\item \texttt{get\_gpu\_count}: 获取 GPU 数量。
	\item \texttt{is\_support\_vulkan}: 检查是否支持 Vulkan。
	\item \texttt{get\_default\_gpu\_index}: 获取默认 GPU 索引。
	\item \texttt{get\_gpu\_info}: 获取指定 GPU 的信息。
	\item \texttt{get\_all\_gpu\_info}: 获取所有 GPU 的信息。
\end{itemize}

\section{CAS\_OCR API的使用}

使用CAS\_OCR API主要分为以下几个步骤:
\begin{enumerate}
	\item 设置工作设备(\texttt{set\_model\_gpu\_support})
	\item 初始化模型(\texttt{init\_model})
	\item 进行预测(\texttt{predict\_validate\_code})
	\item 释放模型(\texttt{release\_model})
\end{enumerate}

其中,\texttt{predict\_validate\_code}函数传递的为cv::Mat,
因此需要先将图像使用OpenCV导入或首先转换为OpenCV的cv::Mat格式。

\section{基于Windows API的Windows GUI程序设计}

使用Microsoft Visual C++ 与 Windows API进行Windows GUI程序开发是微软早期主推的Windows图形界面开发解决方案。其灵活性极强,兼容性极佳广受诸多软件公司的青睐,QQ就是使用MSVC进行开发,虽然QQ目前已经开发了基于Electron的跨平台客户端,但是仅仅在Linux与macOS进行了推广,Windows版本依旧为内测状态。

\subsection{按钮与菜单事件响应}

Windows窗口响应按钮的点击事件以及菜单的点击事件主要通过消息机制。
通过回调函数"LRESULT CALLBACK WndProc(HWND hWnd, UINT message, WPARAM wParam, LPARAM lParam)",其中的"message"参数为接收到的消息类型,"wParam"参数为消息的参数,类型为DWORD双字。

\begin{enumerate}
	\item 我们首先对"WM\_CREATE"进行判断,判断消息类型,如果消息类型为"WM\_COMMAND"则执行下一步判断。
	\item 通过"HIWORD"函数判断"wParam"的高字是否为"BN\_CLICKED",如果命中则进行下一步判断
	\item 通过"LOWORD"函数判断"wParam"的低字是否为按钮对应ID,用于分辨是哪一个按钮触发了onClick事件。
\end{enumerate}

通过上述逻辑进行按钮按下消息的响应。

\subsection{利用Windows Internet API发送网络请求}

"SHMTU\_CAS\_URL.cpp"实现了从指定 URL 获取验证码图像的功能。具体步骤如下:

导入必要的头文件:包括 "framework.h" 和 <wininet.h>这两个均为Windows API,其中framework为VS创建工程自动生成的,其中包含了多个Windows API的头文件,能通过一个头文件一次性导入。

链接 WinINet 库:使用 \#pragma comment(lib, "wininet.lib") 指令将 WinINet 库链接到项目中。

定义 "{get\_captcha\_by\_url}" 函数:

\begin{enumerate}
	\item 函数返回一个 cv::Mat 类型的图像,表示从 URL 获取的验证码图像。
	\item 函数首先指定了验证码图像的 URL。
	\item 使用 InternetOpen 打开一个 Internet 连接。
	\item 使用 InternetOpenUrlA 打开指定的 URL。
	\item 通过循环读取从 URL 返回的数据,并存储到缓冲区中,直到读取完毕。
	\item 使用 cv::imdecode 将读取的图像数据解码为 OpenCV 的 cv::Mat 类型图像。
	\item 最后关闭相关的 Internet 句柄。
\end{enumerate}

调试输出(仅在调试模式下):如果在调试模式下,将输出图像的宽度、高度和通道数。

总体而言,这段代码实现了通过 WinINet 库从指定 URL 获取验证码图像的功能,并将其转换为 OpenCV 中的 cv::Mat 类型。

\subsection{高分辨率屏幕(高分屏)适配}

如今高分屏的价格已经不是多年以前的遥不可及,因此越来越多的人选购高分屏,
此外目前市场上的笔记本均配备高分屏,因此很有必要适配高分屏,让用户获得良好的使用体验。

我通过“\#define ADAPT\_HIGH\_DPI 1”这个宏定义用于控制是否开启DPI适配。

\begin{lstlisting}[caption={DPI缩放},language=C++,label=code:vc_dpi]
	#if ADAPT_HIGH_DPI
	// 告知系统应用程序是 DPI 感知的
	SetProcessDPIAware();
	#endif

	// 获取设备的 DPI
	const HDC screen = GetDC(nullptr);
	const int dpiX =
	GetDeviceCaps(screen, LOGPIXELSX);
	ReleaseDC(nullptr, screen);
	#if ADAPT_HIGH_DPI
	dpi_ratio = static_cast<float>(dpiX) / static_cast<float>(USER_DEFAULT_SCREEN_DPI);
	#endif
\end{lstlisting}

代码请参考\ref{code:vc_dpi},适配步骤如下:

\begin{enumerate}
	\item 告知系统应用程序是 DPI 感知的
	\item 获取设备的 DPI 缩放比例
	\item 根据 DPI 缩放比例计算窗口以及每个控件的位置及大小
\end{enumerate}

效果对比见图\ref{fig:dpilow}与\ref{fig:dpihigh},前者为未适配高DPI,后者为已经适配。
可以观察到未适配版本的标题栏与窗口内部的控件字体显示效果有明显差异,用户体验并不好。
但是适配后每一部分都会非常清晰,因此用户体验极佳。

\begin{figure}
	\centering
	\includegraphics[width=0.9\linewidth]{Resources/Picture/Deploy/Windows/MSVC/dpi_low}
	\caption{4K屏下未适配高DPI}
	\label{fig:dpilow}
\end{figure}

\begin{figure}
	\centering
	\includegraphics[width=0.9\linewidth]{Resources/Picture/Deploy/Windows/MSVC/dpi_high}
	\caption{4K屏下已适配DPI}
	\label{fig:dpihigh}
\end{figure}

\section{运行截图}

\begin{figure}
	\centering
	\includegraphics[width=0.9\linewidth]{Resources/Picture/Deploy/Windows/MSVC/vc_cli}
	\caption{VC CLI}
	\label{fig:vccli}
\end{figure}

\begin{figure}
	\centering
	\includegraphics[width=0.9\linewidth]{Resources/Picture/Deploy/Windows/MSVC/vc_main}
	\caption{VC WinMain}
	\label{fig:vcmain}
\end{figure}

\begin{figure}
	\centering
	\includegraphics[width=0.9\linewidth]{Resources/Picture/Deploy/Windows/MSVC/msvc-win7}
	\caption{Windows 7的运行截图}
	\label{fig:msvc-win7}
\end{figure}
