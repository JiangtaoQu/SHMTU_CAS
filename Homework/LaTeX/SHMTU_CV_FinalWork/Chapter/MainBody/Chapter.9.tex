\newpage
\chapter{Android平台的部署及C++混合编译的探究}
\label{chapter:9}

\section{程序介绍}

上海海事大学验证码识别Demo 是基于前面第\ref{chapter:7}章的VC++代码中的CAS\_OCR.cpp直接移植到Android,然后使用JNI混合编译使用Java调用C++代码。

\section{环境介绍}

Kotlin版本为 1.9.2
为了与Kotlin进行兼容,因此JRE使用Java 1.8

\section{接口设计}

程序核心部分依旧使用OpenCV与NCNN,除此之外还是用JNI相关函数将Android Bitmap转换成OpenCV的cv::Mat,然后进一步进行处理,最后C++输出的std::tuple转换成Java的Object[]。

\section{使用说明}

\begin{enumerate}
	\item 启动程序:点击Luncher上的“上海海事大学验证码识别Demo”,见图\ref{fig:androiddesktop},进入主界面,见图\ref{fig:activitymain}。
	\item 打开图片:“从网络获取验证码”或“本地相册选图”或使用内置图片1和2
	\item 执行OCR:您可以选择使用“CPU识别”或“GPU识别”
\end{enumerate}

\section{运行截图}

\begin{figure}
	\centering
	\includegraphics[width=0.6\linewidth]{Resources/Picture/android_desktop}
	\caption{Android APP 桌面图标}
	\label{fig:androiddesktop}
\end{figure}

\begin{figure}
	\centering
	\includegraphics[width=0.6\linewidth]{Resources/Picture/activity_main}
	\caption{Android APP 主界面}
	\label{fig:activitymain}
\end{figure}


\begin{figure}
	\centering
	\includegraphics[width=0.6\linewidth]{Resources/Picture/app_info}
	\caption{Android APP 信息}
	\label{fig:appinfo}
\end{figure}


