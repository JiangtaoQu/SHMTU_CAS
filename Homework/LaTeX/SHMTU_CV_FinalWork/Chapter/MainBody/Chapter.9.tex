\chapter{Microsoft Windows .Net 平台的部署及混合编译的探究}
\label{chapter:9dotnet}

\section{背景介绍}

.Net 作为一个功能强大的软件开发平台,长期以来一直支持着多种编程语言和各种应用程序类型的开发。然而,随着应用程序的复杂性和规模不断提升,性能优化成为了开发者必须面对的重要挑战。在这种情况下,混合编译技术应运而生,它通过将部分代码编译为机器代码,显著提高了执行效率,为开发者提供了一种有效的优化手段。

微软在当前的技术发展趋势中,已经逐渐将重心转向了C\#的WinForm和WPF技术栈,而传统的VC++则不再是主要推广的焦点。这种转变意味着开发者在构建Windows应用程序时,需要更多地考虑.Net平台上的解决方案。而且WPF不但界面美观还可以完美适配高分屏,无需手动进行适配,节省开发时间。

\section{解决方案探讨}

针对在Microsoft Windows .Net 平台上部署应用程序的需求,本章将介绍两种解决方案,并详细探讨如何结合混合编译技术来提高性能。

\subsection{利用Microsoft ML NuGet包}

Microsoft ML NuGet包为开发者提供了一系列机器学习工具和API,使得在.Net应用程序中部署预训练模型变得更为便捷。通过使用这些工具,开发者可以轻松地集成机器学习功能,而无需具备深厚的C++开发经验。然而,这种方法的一个显著缺点是缺乏灵活性,因为它无法直接利用如OpenCV等第三方库的功能。这意味着,在进行图像处理等相关操作时,开发者可能不得不使用C\# API重新编写代码。

\subsection{通过CLR调用C++代码}

为了弥补Microsoft ML NuGet包在灵活性方面的不足,开发者还可以考虑通过CLR(公共语言运行时)来调用C++编写的代码。通过CLR的桥接作用,C++代码可以无缝集成到.Net应用程序中,从而实现了灵活性和性能的兼顾。当然,这种方法的一个潜在缺点是开发难度较高,要求开发者具备C++开发经验。然而,对于已经熟悉C++的开发者来说,这无疑是一个值得考虑的选择。

此外,这样能够直接调用VC写好的OpenCV图像处理代码以及NCNN推理代码,因此对于我来说,开发成本极低。

\section{工作流程}

\begin{enumerate}
	\item .Net WPF显示界面
	\item C\#下载图片或从本地加载图片
	\item C\#程序通过CLR Connector这个C\#的中间层与CLR进行连接
	\item CLR C++为C\#提供接口
	\item CAS\_OCR.cpp相关逻辑
\end{enumerate}

\section{CLR程序设计}

在CLR(公共语言运行时)程序设计中,实现C\#与C++之间的交互是一个关键环节。由于两种语言使用的数据类型和库函数存在差异,因此需要进行相应的类型转换和接口设计。

\subsection{图像类型转换}

在C\#中,常用的图像类型是System.Drawing.Bitmap,而C++中则通常使用cv::Mat。为了实现两种语言之间的图像数据共享和处理,需要在CLR ClassLibrary中进行类型转换。在"SHMTU\_CAS\_CLR\_ClassLibrary.cpp"和"ImageUtils.cpp"中,我提供了convert\_bitmap\_to\_cv\_mat和cv\_mat\_2\_hbitmap这两个函数,分别用于将C\#中的Bitmap转换为C++中的cv::Mat,以及将cv::Mat转换回C\#中的Bitmap。这两个函数的实现细节涉及到了像素数据的拷贝和格式转换,以确保数据的一致性和正确性。

\subsection{CLR静态库}

值得注意的是,CLR并不支持将动态库编译为静态库。在尝试使用Visual C++编译器进行编译时,会发现"/clr"与"/mt"(用于编译静态库的参数)不能同时使用。因此,我们只能使用"/MDd"参数来编译动态链接库。这也意味着.Net C\#客户端没有静态库版本可供选择。这种限制是由于CLR的运行机制所决定的,它需要在运行时动态加载和链接C++代码,因此无法将其编译为静态库。

\section{CLR Connector程序设计}

虽然CLR类库是用C++编写的,主要用于实现C\#与C++之间的桥梁作用,但在某些情况下,我们仍然需要在C\#中完成一些特定的操作。为了简化这些操作并进一步提高代码的抽象程度,我使用C\#编写了一个CLR Connector的类库。这个类库提供了一系列高级接口和函数,使得在WPF应用程序中只需使用简单的C\#代码就能完成相关业务逻辑。通过这种方式,我们可以将底层的C++代码封装起来,使得C\#开发者无需关心底层的实现细节,只需关注业务逻辑即可。这不仅提高了代码的可读性和可维护性,也降低了开发难度和成本。

\section{运行截图}

WPF程序的主界面见图\ref{fig:wpfmain},在Windows 11中的运行界面见图\ref{fig:dotnetwindows}。

\begin{figure}
	\centering
	\includegraphics[width=0.9\linewidth]{Resources/Picture/Deploy/Windows/DotNet/wpf_main}
	\caption{WPF 主界面}
	\label{fig:wpfmain}
\end{figure}

\begin{figure}
	\centering
	\includegraphics[width=0.9\linewidth]{Resources/Picture/Deploy/Windows/DotNet/DotNet_Windows}
	\caption{C\# .Net WPF在Windows 11中运行}
	\label{fig:dotnetwindows}
\end{figure}
