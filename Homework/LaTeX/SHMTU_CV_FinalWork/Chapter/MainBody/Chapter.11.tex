\chapter{基于Qt C++的跨平台(Windows/macOS/Linux)方案研究}
\label{chapter:11}

\section{项目介绍}

考虑到MSVC和.Net技术栈主要局限于Windows平台,无法实现跨平台部署,我决定探索一种更为灵活和适应性强的解决方案。经过深入研究和比较,我选择了Qt这一成熟且广泛应用的跨平台框架。Qt以其卓越的性能和广泛的兼容性,成为了实现项目跨平台部署的理想选择。

\section{Windows环境下的开发实践}

\subsection{开发环境配置}

为了确保项目的顺利进行,我选择了官方编译的Qt 6.6.2版本(发布日期为2024年2月15日)作为开发基础。由于官方提供的是动态库版本,我决定直接使用这一版本,而未尝试从源代码重新编译静态库版本。因此,最终的Qt应用程序将不包含静态库版本。

\subsection{Qt与MSVC代码差异分析}

在Qt框架与MSVC之间,存在一些显著的区别。Qt主要使用QPixmap对象来处理图像,因此,在将图像数据传递给OpenCV进行处理时,需要进行QPixmap到cv::Mat的转换。这一转换过程在QtOpenCV.cpp文件中得到了详细实现。

此外,Qt采用了信号槽机制来实现事件回调,这与MSVC中的消息机制以及.Net的直接事件绑定方式有所不同。这种机制使得Qt在事件处理上更加灵活和高效。

为了更好地兼容Qt和MSVC,我在"CAS\_OCR\_QT.cpp"文件中对"predict\_validate\_code"函数进行了重载。通过修改函数的参数类型为QPixmap,我们可以直接在Qt环境中调用MSVC中已经实现的C++接口,从而简化了跨平台开发的复杂性。

\subsection{界面展示}

\begin{figure}
	\centering
	\includegraphics[width=0.9\linewidth]{Resources/Picture/Deploy/Qt/Windows/qt_win_main}
	\caption{Qt Windows 主界面}
	\label{fig:qtwinmain}
\end{figure}

\begin{figure}
	\centering
	\includegraphics[width=0.9\linewidth]{Resources/Picture/Deploy/Qt/Windows/qt_windows}
	\caption{Qt Windows 11 运行截图}
	\label{fig:qtwindows}
\end{figure}

\section{适配macOS}

\subsection{开发环境}

\begin{itemize}
	\item XCode
	\item Apple LLVM Clang
	\item Qt 6.6.2
\end{itemize}

\subsection{macOS App Bundle}

macOS App Bundle 是一种特殊的目录结构,用于在 macOS 系统中分发和安装软件。这种目录结构被 macOS 系统识别为一个单独的应用程序实体,用户可以通过双击这个“软件”来运行它。

一个 macOS App Bundle 通常包含以下内容:

\begin{enumerate}
	\item 可执行文件:这是应用程序的主要执行文件,通常以 .app 作为文件扩展名。这个文件包含了程序运行所需的所有代码。
	\item 资源文件:这些文件包含了应用程序运行所需的各种资源,如界面元素、图标、声音文件等。这些资源文件通常被组织在 Resources 子目录中。
	\item 库文件:如果应用程序依赖于特定的库文件,这些库文件也会被包含在 App Bundle 中。这些库文件通常被组织在 Contents/Libraries 或 Contents/Frameworks 子目录中。
	\item 配置文件:应用程序的配置文件也可能被包含在 App Bundle 中。这些文件通常被组织在 Contents/Resources 或 Contents/Preferences 子目录中。
	\item 元数据:App Bundle 还可能包含一些元数据文件,如 Info.plist,它包含了应用程序的元信息,如名称、版本、图标等。
\end{enumerate}

在 macOS 中,"*.app" 目录实际上就是一个特殊的目录结构,它遵循了 macOS 的文件系统规范和命名约定。当你双击一个 .app 文件时,macOS 系统会解析这个目录结构,找到可执行文件并运行它。同时,系统也会根据 App Bundle 中的其他文件和目录来加载资源、配置和库文件,以确保应用程序能够正常运行。

总的来说,macOS App Bundle 是一种方便用户在 macOS 系统上安装和运行软件的方式。它使得应用程序的分发和安装变得更加简单和直接,同时也为应用程序提供了更好的集成和兼容性。

\subsection{生成macOS的icns文件}

macOS的图标不同于Windows平台的iso,为苹果平台专用的icns文件,因此我编写了一个脚本用于自动划分图片尺寸以及生成icns文件,具体请参考\url{Deploy/QT/SHMTU_CAS_OCR_Demo_QT/Resources/shmtu_logo_black/mkicns.sh}。

\subsection{CMake打包CheckPoint}

CMake中包含了脚本,能够在构建.app目录的同时,将Checkpoint目录自动拷贝到Resources中,因此我在C++代码中添加了"\_\_APPLE\_\_"的编译器宏指令,用于判断是否为macOS,如果为macOS,预先设定好的Checkpoint搜索路径将与其他操作系统不同。

\begin{figure}
	\centering
	\includegraphics[width=0.9\linewidth]{Resources/Picture/Deploy/Qt/macOS/mac_bundle}
	\caption{CMake macOS打包权重文件}
	\label{fig:macbundle}
\end{figure}

\subsection{界面展示}

\begin{figure}
	\centering
	\includegraphics[width=0.9\linewidth]{Resources/Picture/Deploy/Qt/macOS/qt_mac_main}
	\caption{Qt macOS 主界面}
	\label{fig:qtmacmain}
\end{figure}

\begin{figure}
	\centering
	\includegraphics[width=0.9\linewidth]{Resources/Picture/Deploy/Qt/macOS/mac}
	\caption{Qt macOS 14 运行截图}
	\label{fig:mac}
\end{figure}

\section{适配Linux}

\subsection{开发环境}

\begin{enumerate}
	\item Ubuntu 23.10
	\item gcc 12.3.0
	\item Qt 6.6.2
	\item Qt Creator 12.0.2
	\item CLion
	\item CMake
\end{enumerate}

\subsection{Linux特殊处理}

Linux的特殊处理在于需要手动导入"feature.h"这个头文件才能实现宏判断当前操作系统。

\subsection{界面展示}

\begin{figure}
	\centering
	\includegraphics[width=0.9\linewidth]{Resources/Picture/Deploy/Qt/Linux/qt_linux_main}
	\caption{Qt Linux 主界面}
	\label{fig:qtlinuxmain}
\end{figure}

\begin{figure}
	\centering
	\includegraphics[width=0.9\linewidth]{Resources/Picture/Deploy/Qt/Linux/linux}
	\caption{Qt Ubuntu 23.10 在KDE桌面环境下的运行截图}
	\label{fig:linux}
\end{figure}

