\chapter{基于Qt C++的跨平台(Windows/macOS/Linux)方案研究}
\label{chapter:11}

\section{介绍}

因为无论是MSVC还是.Net均为Windows下的技术栈,无法部署到其他平台,因此我决定开发一个跨平台的方案,因此我选择了Qt这个成熟的跨平台方案。

\section{Windows下进行开发}

\subsection{开发环境}

使用官方编译的Qt6.6.2(2024年2月15日发布)版本,因为官方编译的为动态库版本,因此我没有重新从src编译静态库版本,因此最终Qt程序并没有静态库版本。

\subsection{Qt与MSVC版本代码的区别}

Qt程序与MSVC主要区别在于,Qt使用QPixmap,因此只需要将QPixmap转换为cv::Mat即可,此外Qt使用信号槽机制来进行事件回调,不同于VC的消息机制,以及.Net直接绑定响应事件。

QPixmap与cv::Mat的转换请参考\url{Deploy/QT/SHMTU_CAS_OCR_Demo_QT/QtOpenCV.cpp}。

此外我在"CAS\_OCR\_QT.cpp"文件中重载了"predict\_validate\_code"函数,将第一个参数变为QPixmap类型。
这样就可以方便地在Qt中直接调用MSVC已经完成的C++接口。

\subsection{界面展示}

\section{适配macOS}

\subsection{开发环境}

\begin{itemize}
	\item XCode
	\item Apple LLVM Clang
	\item Qt 6.6.2
\end{itemize}

\subsection{生成macOS的icns文件}

\subsection{打包CheckPoint}

\subsection{界面展示}

\section{适配Linux}

\subsection{开发环境}

\begin{enumerate}
	\item Ubuntu 23.10
	\item gcc 12.3.0
	\item Qt 6.6.2
	\item Qt Creator 12.0.2
	\item CLion
	\item CMake
\end{enumerate}

\subsection{界面展示}
