\chapter{模型的搭建}
\label{chapter:4}

\section{ResNet网络}

\subsection{介绍}

ResNet\cite{He_2016_CVPR},即残差网络,是深度学习领域里程碑式的卷积神经网络架构之一。它在2015年由微软研究院的杰出研究者Kaiming He及其团队首次提出,并在当年的ImageNet图像识别挑战赛中大放异彩,一举夺得了冠军。ResNet的成功并非偶然,它背后蕴含着对深度学习本质的深刻理解与创新设计。

ResNet的核心贡献在于引入了残差学习(residual learning)的崭新概念。在传统的深度神经网络中,随着网络层数的不断增加,训练过程中的梯度消失和模型退化问题日益严重,这极大地限制了网络性能的进一步提升。为了解决这一难题,ResNet巧妙地设计了残差块(residual block),这是一种特殊的网络结构单元,能够使得网络在训练过程中更加关注于残差部分的学习。

具体来说,残差块通过引入跳跃连接(skip connection),将输入直接加到卷积层的输出上,从而形成了残差连接。这种设计不仅有效地缓解了梯度消失的问题,使得梯度能够更加顺畅地回流到较早的层,而且还能够避免模型退化现象的发生。换句话说,即使在网络层数非常深的情况下,ResNet依然能够保持良好的性能表现。

正是由于ResNet在残差学习和残差块设计方面的卓越贡献,它才能够训练出更深、更强大的神经网络模型。这一突破性的成果不仅推动了深度学习领域的快速发展,而且为后来的研究者们提供了宝贵的启示和借鉴。如今,ResNet已经成为计算机视觉领域中最受欢迎的神经网络架构之一,被广泛应用于图像分类、目标检测、语义分割等各种任务中。

\subsection{残差学习}

在深度神经网络中,通常我们认为随着网络层数的不断增加,模型的复杂度也会相应提升,理应能够更好地拟合训练数据。然而,实践却表明,当网络层数增加到一定程度后,训练误差反而会出现上升的现象,这是由于梯度消失和模型退化这两个棘手问题所导致的。梯度消失具体是指在反向传播过程中,梯度值逐层递减,变得极其微小,以至于网络权重几乎无法得到有效更新。而模型退化则表现为随着网络层数的加深,网络的性能不仅没有得到提升,反而出现了下降的情况。

为了解决这些问题,ResNet引入了残差学习的思想,为深度学习领域带来了革命性的突破。残差学习的核心思想在于学习输入与输出之间的残差函数,而非直接学习从输入到输出的复杂映射。具体来说,对于一个网络层,设其输入为$x$,输出为$H(x)$,则残差函数定义为$F(x) = H(x) - x$。在ResNet中,网络通过学习这个残差函数$F(x)$,并将其与输入$x$相加,得到最终的输出$H(x) = F(x) + x$。这种设计巧妙地使得网络在训练过程中更加关注于残差部分的学习,从而有效地避免了梯度消失和模型退化问题。通过残差学习,ResNet成功实现了对网络深度的有效扩展,大幅提升了模型的性能。

\subsection{残差块}

为了实现残差学习,ResNet精心设计了一种特殊的网络结构单元,即残差块。残差块由多个卷积层、批量归一化层和ReLU激活函数组成,形成了一个强大的计算单元。在每个残差块的最后,通过引入一个跳跃连接或shortcut connection,将输入直接加到输出上,实现了残差连接。这种跳跃连接的设计巧妙地使得梯度能够直接回流到较早的层,从而有效地缓解了梯度消失问题。同时,残差块中的批量归一化层也有助于加速训练过程并提高模型的泛化能力。

在ResNet中,根据网络深度的不同需求,设计了两种主要的残差块:基本残差块和瓶颈残差块。基本残差块主要由两个3x3的卷积层组成,适用于构建相对较浅的网络结构。而瓶颈残差块则采用了更为复杂的结构设计,包括1x1、3x3和1x1三个卷积层的组合。其中,1x1的卷积层被用于降低和恢复维度,以减少计算量并提高计算效率。这种设计使得瓶颈残差块更加适用于构建更深的网络结构。

\subsection{网络架构}

ResNet的网络架构由多个精心设计的残差块堆叠而成,形成了一个深度强大且高效的网络模型。根据不同的应用场景和性能需求,ResNet有多个变体可供选择,如ResNet-18、ResNet-34、ResNet-50、ResNet-101和ResNet-152等。这些变体主要在网络深度和残差块类型上有所区别,以满足不同任务的需求。例如,对于较为简单的图像分类任务,可以选择相对较浅的ResNet-18或ResNet-34;而对于更复杂的任务如目标检测或语义分割,则可能需要选择更深的ResNet-50、ResNet-101或ResNet-152以获得更好的性能。

除了标准的ResNet架构外,研究者们还不断探索和改进ResNet的设计,提出了许多改进版本。例如,ResNeXt通过引入分组卷积的思想来扩展ResNet的宽度;SE-ResNet则通过引入注意力机制来增强模型的特征表示能力;而EfficientNet则通过一种复合缩放策略来同时优化网络的深度、宽度和分辨率。这些改进版本在保持ResNet核心思想的同时,进一步提升了网络的性能,推动了深度学习领域的发展。

\subsection{应用与影响}

由于ResNet出色的性能和灵活性,它在计算机视觉领域得到了广泛的应用和认可。除了图像分类任务外,ResNet还被成功应用于目标检测、语义分割、人脸识别、姿态估计等多个具有挑战性的任务中。此外,ResNet的思想也被借鉴到其他类型的神经网络设计中,如循环神经网络和自然语言处理等领域,为这些领域的发展带来了新的启示和机遇。

ResNet的成功对深度学习领域产生了深远的影响。它证明了通过精心设计网络结构和引入新的训练策略,我们可以训练出更深、更强大的神经网络来处理复杂的任务。同时,ResNet也为后续的研究提供了宝贵的经验和启示,推动了深度学习领域的进一步发展和创新。如今,ResNet已经成为深度学习领域中最受欢迎的神经网络架构之一,为人工智能的发展做出了重要贡献。

\section{ResNet网络的应用}

ResNet网络因其强大的特征提取能力而广受欢迎,它经常被用作其他复杂网络结构的基础,即所谓的Back Bone。PyTorch深度学习框架为研究者们提供了极大的便利,内置了ResNet网络模型,这使得我们无需从零开始手动搭建网络结构。

\begin{enumerate}
	\item 为了使用PyTorch内置的ResNet模型,我们首先需要从\textit{torchvision.models}模块中导入预训练好的网络模型。
	\item 接下来,我们需要对模型的最后一层全连接层进行调整,以便适应我们的特定分类任务需求。
\end{enumerate}

关于如何使用ResNet网络的具体代码,请参考我的训练脚本:
\url{Train/SHMTU\_CAS\_OCR\_RESNET/src/classify/step/train.py}

\subsection{等于号的分类任务}

在符号识别领域,等于号的分类任务显得相对简单。由于等于号本身只有两类,即汉字“等于”和符号“=”,这两者在视觉形态上存在着明显的差异。汉字“等于”由两个独立的字符组成,而符号“=”则是一条直线。这种明显的形态差异使得我们不需要采用过于复杂的模型来进行分类。

因此,我们选择使用ResNet-18这一轻量级的卷积神经网络来进行这一2分类任务。ResNet-18以其高效的性能和良好的泛化能力在图像分类任务中得到了广泛的应用。通过利用ResNet-18的强大特征提取能力,我们可以有效地从输入的图像中提取出区分汉字“等于”和符号“=”的关键特征,从而实现高精度的分类。

在训练过程中,我们将准备大量的带有标签的等于号图像数据,这些数据将用于训练ResNet-18模型。通过不断地优化模型的参数,我们可以使得模型对于这两类等于号的识别能力逐渐提高。最终,我们可以得到一个能够准确区分汉字“等于”和符号“=”的ResNet-18模型,为后续的符号识别任务提供有力的支持。

\subsection{运算符的分类任务}

运算符的分类在符号识别领域中扮演着至关重要的角色,它涵盖了数学和汉字领域中多种常见的运算符。这不仅仅是一个单纯的图像识别任务,更是一个涉及到语义理解和上下文分析的综合任务。在本分类任务中,我们主要关注以下六类运算符:

\begin{enumerate}
	\item 加号(+):这是数学运算中的基石,表示两个数值的相加。加号在日常生活、学术研究以及各类计算中无处不在,其准确识别对于任何符号识别系统来说都至关重要。
	\item 减号(-):与加号相对应,减号用于表示两个数值之间的差值。在数学表达式和计算过程中,减号的准确识别同样具有不可或缺的地位。
	\item 乘号(×):乘号在数学中用于表示两个数值的相乘。尽管在现代计算中,乘号有时被省略或用星号(*)代替,但乘号作为标准的乘法运算符,其地位仍不可撼动。
	\item 汉字“加”:在中文数学表达式中,汉字“加”被广泛用于表示加法运算。与数学符号加号(+)不同,汉字“加”不仅提供了数学运算的信息,还蕴含了中文的语义和语境,使得数学表达式更易于理解和解读。
	\item 汉字“减”:在中文数学语境下,汉字“减”同样扮演着重要的角色,用于表示减法运算。对于包含减法的数学语句来说,汉字“减”的准确识别是理解整个语句意义的关键。
	\item 汉字“乘”:与数学符号乘号(×)功能相似,汉字“乘”在中文中也被用于表示乘法运算。与数学符号不同,汉字“乘”提供了更多的上下文信息,使得数学表达式在中文语境下更加清晰和易于理解。
\end{enumerate}

这六类运算符,无论是数学符号还是汉字,都在数学和中文表达中占据了举足轻重的地位。它们的准确识别对于构建一个高效、准确的符号识别系统来说至关重要。尽管这六类运算符在形态和语义上各有特点,但考虑到ResNet-18模型在图像分类任务中的出色表现以及其对复杂特征的强大捕捉能力,我们依然选择使用ResNet-18模型进行这六类运算符的分类任务。通过精心设计和调整模型的训练策略,我们有信心能够训练出一个能够准确区分这六类运算符的识别系统,为后续的符号识别任务提供有力的支持。

\subsection{阿拉伯数字的分类任务}

阿拉伯数字的分类任务在符号识别领域中具有举足轻重的地位。与运算符分类任务不同,阿拉伯数字的分类涵盖了从0到9的十个独立类别,每个数字都具有其独特的形态和语义含义。这些数字在日常生活中扮演着至关重要的角色,不仅用于数学计算,还广泛应用于时间表示、文档编号、电话号码等各个领域。

要实现对阿拉伯数字的准确分类,我们需要构建一个精细且稳健的识别系统。这个系统需要具备出色的区分能力,特别是对于那些形态相似的数字,如6和9,它们之间的细微差别往往决定了识别的准确性。此外,由于在实际应用中,可能会遇到裁剪不完整或书写风格各异的数字图像,如数字7在裁剪不完整时可能与带衬线的数字1混淆,这对识别系统提出了更高的要求。

为了应对这些挑战,我们不仅需要收集大量带有标签的阿拉伯数字图像数据,还需要选择一个具有强大特征提取能力的模型。尽管ResNet-18模型在许多任务中表现出色,但由于其对于复杂和细微特征识别的局限性,我们可能需要考虑一个更高级的模型。

在综合考虑模型的性能、体积和准确率之后,我们选择了ResNet-34模型进行阿拉伯数字的分类任务。ResNet-34模型通过其更深的网络结构和更强的特征提取能力,能够更有效地捕捉和区分阿拉伯数字的关键特征,特别是在处理形态相似或裁剪不完整的数字时。同时,我们也对模型进行了优化,以确保在保持较高准确率的同时,尽可能地减小体积和计算复杂度,以满足移动端部署的需求。

通过利用ResNet-34模型进行训练,我们可以期望构建一个既高效又准确的阿拉伯数字识别系统。这将为各种依赖数字识别的应用场景提供强大的支持,如文档自动化处理、表格识别、验证码识别等。同时,这一技术也将为相关领域的研究和开发提供新的思路和方法。
